\documentclass[12pt]{article}
\usepackage{mathptmx}
\usepackage[utf8]{inputenc}
\usepackage{cmap}
\usepackage{mathtext}
\usepackage{array}
\usepackage{booktabs}
\usepackage{multirow}
\usepackage[russian,english]{babel}
\usepackage{amsmath,amsfonts,amssymb,amsthm,mathtools} % AMS
\usepackage{icomma}
\usepackage[T2A]{fontenc}
\usepackage[left=10mm, top=20mm, right=18mm, bottom=15mm, footskip=10mm]{geometry}
\usepackage{indentfirst}
\usepackage{euscript}	 % Шрифт Евклид
\usepackage{mathrsfs} % Красивый матшрифт
\usepackage{amsmath,amssymb}
\usepackage[italicdiff]{physics}
\usepackage{graphicx}
\graphicspath{{images/}}
\DeclareGraphicsExtensions{.pdf,.png,.jpg}
\usepackage{wrapfig}
\usepackage{subcaption}
\usepackage{caption}
\usepackage{hyperref}
\usepackage[rgb]{xcolor}
\hypersetup{				% Гиперссылки
    colorlinks=true,       	% false: ссылки в рамках
	urlcolor=blue          % на URL
}
\captionsetup[figure]{name=Рисунок}
\captionsetup[table]{name=Таблица}


% Определяем команду для строки оглавления с отточками
\newcommand{\tocline}[2]{%
  \noindent\makebox[0pt][l]{#1}\hfill
  \leaders\hbox to 1.5mm{\hss.\hss}\hfill
  \makebox[0pt][r]{#2}\par
}

% Альтернативный стиль (более надёжный, работает всегда):
\newcommand{\tocentry}[2]{%
  \noindent
  \parbox[t]{0.8\textwidth}{\raggedright #1}%
  \parbox[t]{0.15\textwidth}{\raggedleft #2}%
  \par
  \vspace{-1.2\baselineskip}
  \noindent\leaders\hbox to 1.5mm{\hss.\hss}\hfill\mbox{}%
  \par
  \vspace{0.8\baselineskip}
}

% Но самый простой и надёжный способ — через \dotfill:
\newcommand{\tocitem}[2]{%
  \noindent #1 \dotfill\ #2\par
}


\begin{document}

\thispagestyle{empty}

\begin{center}

% Верхняя часть титульного листа
\large
Министерство науки и высшего образования Российской Федерации \\
\large
ФЕДЕРАЛЬНОЕ ГОСУДАРСТВЕННОЕ АВТОНОМНОЕ \\
ОБРАЗОВАТЕЛЬНОЕ УЧРЕЖДЕНИЕ ВЫСШЕГО ОБРАЗОВАНИЯ \\
\large
«МОСКОВСКИЙ ФИЗИКО-ТЕХНИЧЕСКИЙ ИНСТИТУТ \\
(НАЦИОНАЛЬНЫЙ ИССЛЕДОВАТЕЛЬСКИЙ УНИВЕРСИТЕТ)» \\
(МФТИ, Физтех)

\vspace{4cm}

% Основная информация
\large
КАФЕДРА ЭЛЕКТРОНИКИ

\vspace{0.7cm}

\Large
ОТЧЕТ \\
ПО ЛАБОРАТОРНОЙ РАБОТЕ

\vspace{0.7cm}

\large
\textbf{МЕТОДЫ ПОЛУЧЕНИЯ ВЫСОКОГО ВАКУУМА}

\vfill

% Блок с подписями
\noindent
Работу выполнил \hfill \underline{\hspace{6cm}} Д.А.Воронин Б04-407 \\
\hfill \small{(подпись, дата)}\par
\hfill \underline{\hspace{6cm}} В.С.Анисимов Б04-407\\
\hfill \small{(подпись, дата)}\par
\hfill \underline{\hspace{6cm}} М. .Кручинин Б04-407\\
\hfill \small{(подпись, дата)}\par

\vspace{3cm}

\noindent
Работу принял, оценка \hfill \underline{\hspace{6cm}} \\
\hfill \small{(подпись, дата, оценка)}

\vfill

% Нижняя часть
Долгопрудный 2025

\end{center}

\newpage
\section*{СОДЕРЖАНИЕ}

\vspace{1em}

\tocitem{Цель работы}{3}
\tocitem{Теоретические свеедния}{3}
\tocitem{Лабораторная установка}{5}
\tocitem{Ход работы}{7}
\tocitem{\hspace{1em}4.1. Форвакуумный насос}{7}
\tocitem{\hspace{1em}4.2.Турбомолекулярный насос}{9}
\tocitem{\hspace{1em}4.4. Поток воздуха через диафрагму}{9}
\tocitem{\hspace{1em}4.5.Производительность турбомолекулярного насоса}{11}
\tocitem{\hspace{1em}4.6.Определение объема вакуумной системы}{11}
\tocitem{Анализ результатов}{13}
\tocitem{Выводы}{14}
\tocitem{Приложение}{14}


\newpage
\section{ЦЕЛЬ РАБОТЫ}

\begin{enumerate}
    \item Ознакомиться с принципами работы вакуумной техники: пластинчато-роторного насоса, турбомолекулярного насоса, ионизационного, ёмкостного и терморезистивного вакуумметров.
    \item Ознакомиться с методами вакуумных расчётов, найти зависимость величины газового потока в системе от давления.
    \item Определить производительность турбомолекулярного насоса.
    \item Рассчитать объем рабочей камеры.
\end{enumerate}

\section{ТЕОРЕТИЧЕСКИЕ СВЕДЕНИЯ}

\subsection*{Основные понятия, используемые в работе}

% В вакуумной технике используются следующие ключевые параметры:

\begin{itemize}
\item \textbf{Поток газа} $Q$ — количество газа, проходящего через поперечное сечение трубопровода за единицу времени
\item \textbf{Проводимость} $U$ — способность вакуумной системы пропускать газ:
\begin{equation}
U = \frac{Q}{(P_2 - P_1)}
\label{eq:conductance}
\end{equation}
\item \textbf{Быстрота действия насоса} $S_n$ — объём газа, поступающего в насос в единицу времени при давлении $P_1$:
\begin{equation}
S_n = -\frac{dV_n}{d\tau}\bigg|_{P_1}
\label{eq:pump_speed}
\end{equation}
% \item \textbf{Быстрота откачки объема} $S_0$ — скорость откачки со стороны откачиваемого объекта
\end{itemize}

\subsection*{Основное уравнение вакуумной техники}

Для системы, состоящей из откачиваемого объема, трубопровода и насоса, справедливо соотношение:

\begin{equation}
\frac{1}{S_0} = \frac{1}{U} + \frac{1}{S_n}
\label{eq:main_vacuum}
\end{equation}

где $U$ — проводимость трубопровода, соединяющего насос с откачиваемым объемом.

\subsection*{Режимы течения газа}

\textbf{Вязкостный режим} (ламинарное течение) характерен для высоких давлений. Проводимость круглого трубопровода:

\begin{equation}
U_{\text{тв}} = \frac{\pi r_0^4 (P_2 + P_1)}{16\eta l}
\label{eq:viscous_flow}
\end{equation}

\textbf{Молекулярный режим} возникает при низких давлениях. Проводимость круглого трубопровода:

\begin{equation}
U_{\text{мол}} = \frac{\pi d^3 \langle v \rangle}{12l}
\label{eq:molecular_flow}
\end{equation}

где $\langle v \rangle = \sqrt{\frac{8RT}{\pi\mu}}$ — средняя тепловая скорость молекул.

\subsection*{Проводимость диафрагмы}

Для молекулярного течения через отверстие проводимость определяется формулой:

\begin{equation}
U_{\text{отв}} = \frac{S \langle v \rangle}{4} = 91 \cdot d^2 \quad [\text{л/с}]
\label{eq:orifice_conductance}
\end{equation}

где $d$ выражено в см.

\subsection*{Уравнение откачки}

Процесс откачки описывается дифференциальным уравнением:

\begin{equation}
\frac{dP}{dt} = -\frac{S(P) \cdot P}{V}
\label{eq:pumping_equation}
\end{equation}

При постоянной быстроте действия $S = const$ решение имеет вид:

\begin{equation}
P(t) = P_0 \cdot \exp\left(-\frac{S}{V} \cdot t\right)
\label{eq:exponential_pumping}
\end{equation}

В реальных условиях, когда $S$ зависит от давления, используется линеаризованная форма:

\begin{equation}
\ln P(t) = \ln P_0 - \frac{S}{V} \cdot t
\label{eq:linearized_pumping}
\end{equation}

\subsection*{Методы измерения быстроты действия}

\textbf{Метод постоянного объема} основан на измерении зависимости $P(t)$:

\begin{equation}
S_n = \frac{V}{\tau} \ln \frac{P_0}{P_1}
\label{eq:constant_volume}
\end{equation}

\textbf{Метод постоянного давления} используется для высокопроизводительных насосов:

\begin{equation}
S_n = \frac{Q}{P_n}
\label{eq:constant_pressure}
\end{equation}

где $Q$ — измеренный поток газа, напускаемого в систему.

\subsection*{Критерий режима течения}

Режим течения газа определяется соотношением между длиной свободного пробега молекул $\lambda$ и характерным размером системы $d$:

\begin{itemize}
\item Вязкостный режим: $\lambda \ll d$
\item Молекулярный режим: $\lambda \gg d$
\item Переходный режим: $\lambda \sim d$
\end{itemize}

Длина свободного пробега вычисляется по формуле:

\[\lambda =\frac{kT}{\sqrt{2}\pi d^{2}_{\text{М}}P}\]

\section{ЛАБОРАТОРНАЯ УСТАНОВКА}

Лабораторная установка предназначена для ознакомления с основными приборами вакуумной 
техники: насосами, манометрами, измерителями расхода газа. Схема установки представлена на рисунке 1.


\begin{figure}[h]
    \centering
    \includegraphics[width=\textwidth]{p1.PNG}
    \caption{Схема лабораторной установки}
    \label{fig:vac}
\end{figure}

На схеме обозначены: \\
\begin{itemize}
\item$B_1$ - вакуумметр ёмкостной
\item$B_2$ - вакуумметр терморезисторный
\item$B_3$ - вакуумметр ионизационный
\item$K_1$ - кран турбомолекулярного насоса
\item$K_3$ - высоковакуумная заслонка
\item$K_4$ - форвакуумная заслонка
\item$K_2, K_7$ - коммутационные краны
\item Д - диафрагма
\item FC - регулятор газового потока (flow controller)
\item ТМН - турбомолекулярный насос
\item ФВН - форвакуумный насос
\end{itemize}

\begin{table}[htbp]
\centering
\caption{Характеристики форвакуумного насоса}
\label{tab:system_params}
\begin{tabular}{|l|c|c|c|c|}
\hline
\textbf{Параметр} & \textbf{5} & \textbf{9} & \textbf{14} & \textbf{18} \\
\hline
Фланец (вход) & \multicolumn{4}{c|}{DN 25 ISO-KF} \\
\hline
Фланец (выход) & \multicolumn{4}{c|}{DN 25 ISO-KF} \\
\hline
Скорость откачки, м$^3$/ч & 5 & 9 & 14 & 18 \\
\hline

Предельное давление (без газового балласта), мбар & \multicolumn{4}{c|}{$5 \cdot 10^{-4}$} \\
\hline
Предельное давление (с газовым балластом), мбар & \multicolumn{4}{c|}{$1 \cdot 10^{-2}$} \\
\hline
Ёмкость масляной системы, л & 0,83 & 0,95 & 0,95 & 0,98 \\
\hline
\end{tabular}
\end{table}

\begin{table}[htbp]
\centering
\caption{Технические характеристики масс-спектрометра ВЗ}
\label{tab:mass_spec_params}
\begin{tabular}{|p{6cm}|p{8cm}|}
\hline
\textbf{Параметр} & \textbf{Значение} \\
\hline
Диапазон измеряемых масс & 1--100 или 1--200 а.е.м. \\
\hline
Длина фильтра & 4" (100 мм) \\
\hline
Детектор & чаша Фарадея или \\ 
          & чаша Фарадея с УВЭ (умножителем вторичных электронов) \\
\hline
Катоды & Вольфрамовый или \\ 
       & иридиевый, покрытый окисью тория \\
\hline
Максимальное рабочее давление & $1 \times 10^{-4}$ торр ($1{,}3 \times 10^{-4}$ мбар) \\
\hline
Чувствительность ионного источника & $2 \times 10^{-4}$ А/мбар \\
\hline
Мин. определяемое парциальное давление & $2 \times 10^{-11}$ торр ($2{,}6 \times 10^{-11}$ мбар) — чаша Фарадея \\
(накопление в течение 300 мс, 3 среднеквадратических отклонения) & $5 \times 10^{-14}$ торр ($6{,}7 \times 10^{-14}$ мбар) — умножитель \\
\hline
\end{tabular}
\end{table}

\newpage

\section{ХОД РАБОТЫ}




\subsection{Форвакуумный насос}
После ознакомления с установкой был запущен форвакуумный насос, ёмкостной (В1) и терморезисторный (В2) вакуумметры и регулятор газового потока, изначально установленный на 0 см$^3$/мин.\par
Затем была получена зависимость давления от времени при откачке системы форвакуумным насосом (рис. 2) и зависимость давления от величины газового потока (рис. 3) 
Получена зависимость производительности насоса $S = Q/B$ от входного давления.



\begin{table}[h]
\centering
\begin{minipage}{0.48\textwidth}
\centering
\caption{Данные при увеличении $Q_{\text{set}}$ от 5 до 100}
\label{tab:increase}
\begin{tabular}{|c|c|c|c|}
\hline
$Q_{\text{set}}$ & $Q_{\text{факт}}$ & $B2$ & $S$ \\
\hline
5 & 6.9 & 0.164 & 42.07 \\
10 & 12.3 & 0.242 & 50.83 \\
15 & 17.47 & 0.308 & 56.72 \\
20 & 22.62 & 0.371 & 60.97 \\
25 & 27.74 & 0.431 & 64.36 \\
30 & 32.83 & 0.488 & 67.27 \\
35 & 37.87 & 0.545 & 69.49 \\
40 & 42.96 & 0.599 & 71.72 \\
45 & 48.00 & 0.655 & 73.28 \\
50 & 53.00 & 0.709 & 74.75 \\
55 & 58.12 & 0.764 & 76.07 \\
60 & 63.15 & 0.818 & 77.20 \\
65 & 68.19 & 0.874 & 78.02 \\
70 & 73.20 & 0.915 & 80.00 \\
75 & 78.27 & 0.944 & 82.91 \\
80 & 83.26 & 0.973 & 85.57 \\
85 & 88.25 & 1.000 & 88.25 \\
90 & 93.28 & 1.060 & 88.00 \\
95 & 98.37 & 1.120 & 87.83 \\
100 & 103.40 & 1.180 & 87.63 \\
\hline
\end{tabular}
\end{minipage}
\hfill
\begin{minipage}{0.48\textwidth}
\centering
\caption{Данные при уменьшении $Q_{\text{set}}$ от 100 до 0}
\label{tab:decrease}
\begin{tabular}{|c|c|c|c|}
\hline
$Q_{\text{set}}$ & $Q_{\text{факт}}$ & $B2$ & $S$ \\
\hline
95 & 98.32 & 1.12 & 87.79 \\
90 & 93.23 & 1.07 & 87.13 \\
85 & 88.24 & 1.01 & 87.37 \\
80 & 83.12 & 0.975 & 85.25 \\
75 & 78.12 & 0.946 & 82.58 \\
70 & 73.06 & 0.917 & 79.67 \\
65 & 68.03 & 0.878 & 77.48 \\
60 & 62.99 & 0.824 & 76.44 \\
55 & 57.94 & 0.769 & 75.34 \\
50 & 52.89 & 0.713 & 74.18 \\
45 & 47.80 & 0.658 & 72.64 \\
40 & 42.79 & 0.602 & 71.08 \\
35 & 37.70 & 0.545 & 69.17 \\
30 & 32.65 & 0.487 & 67.04 \\
25 & 27.58 & 0.428 & 64.44 \\
20 & 22.46 & 0.368 & 61.03 \\
15 & 17.36 & 0.308 & 56.36 \\
10 & 12.20 & 0.235 & 51.91 \\
5 & 6.90 & 0.159 & 43.40 \\
0 & 1.07 & 0.033 & 32.42 \\
\hline
\end{tabular}
\end{minipage}
\end{table}

\begin{figure}[h]
    \centering
    \includegraphics[width=\textwidth]{p2.png}
    \caption{Зависимость B2 от $Q_{Set}$}
    \label{fig:vac}
\end{figure}

\begin{figure}[h!]
	\centering
	\includegraphics[width=21cm]{p3.PNG}
	\caption{График зависимости давления в форвакуумной части от времени}
	\label{fig:Holl2}
\end{figure}

\begin{figure}[h!]
	\centering
	\includegraphics[width=21cm]{p7.PNG}
	\caption{График зависимости
быстроты действия форвакуумного
насоса от давления}
	\label{fig:Holl2}
\end{figure}

\newpage

\begin{figure}[h!]
	\centering
	\includegraphics[width=21cm]{p5.PNG}
	\caption{ График зависимости давления в высоковакуумной части от времени}
	\label{fig:Holl2}
\end{figure}

\subsection{Параметры установки}

\begin{itemize}
    \item Диаметр диафрагмы: $d = 100$ мкм $= 0.01$ см
    \item Газ: воздух (молярная масса $M = 29$ г/моль)
    \item Температура: $T = 293$ К
\end{itemize}

\subsection{Турбомолекулярный насос}
Был включён турбомолекулярный насос и ионизационный вакуумметр (В3), после чего была получена зависимость давления в высоковакуумной части от времени при откачке и при потоке газа через диафрагму.
Появление скачков на некоторых графиках связано с тем, что ионизационный вакуумметр переключает режим работы, повышая накал нити при уменьшении давления.

\subsection{Поток воздуха через фиафрагму}

Определим, можно ли считать течение газа через диафрагму молекулярным. Для этого оценим длину свободного пробега молекул:
\begin{center}
$\lambda = \frac{kT}{\sigma P} \approx 1,24$ м,
\end{center}
где $k = 1,38 \cdot 10^{-23}$ Дж/К -- постоянная Больцмана\\
    %   $T \approx 293$ К – комнатная температура\\
       $\sigma = 62,5 \cdot 10^{-20} $ м$^2$ -- среднее эффективное сечение рассеяния для воздуха\\
       $P \approx 3 \cdot 10^{-5}$ Торр -- максимальное давление в высоковакуумной части системы\\

Видно, что $d \ll \lambda$, поэтому течение газа через диафрагму можно считать молекулярным. Следовательно, справедлива формула нахождения молекулярного потока через диафргаму:
\begin{center}
$Q = S\sqrt{\frac{RT}{2\pi\mu}}(P_2-P_3)$
\end{center}
где $P_2, P_3$ - давления на В2 и В3 соответственно\\
$S = \frac{\pi d^2}{4}$ - площадь отверстия в диафрагме\\
$\mu$ - молярная масса воздуха.

Подставив величины, получаем, что 
\begin{center}
$Q \approx 8.9\cdot10^{-4}(P_2-P_3)$ л/с (давление выражено в Торрах)
\end{center}
График зависимости величины потока через диафрагму от времени изображён на рисунке:

\begin{figure}[h!]
	\centering
	\includegraphics[width=21cm]{p4.PNG}
	\caption{График зависимости потока воздуха через диафрагму от времени}
	\label{fig:Holl2}
\end{figure}

% \begin{figure}[h!]
% 	\centering
% 	\includegraphics[width=20cm]{p7.PNG}
% 	\caption{График зависимости быстроты действия форвакуумного насоса от давления}
% 	\label{fig:Holl2}
% \end{figure}
\newpage
\subsection{Производительность турбомолекулярного насоса}
Рассмотрим модель потока через турбомолекулярный насос. В установившемся режиме выполняется баланс:
\begin{center}
$P_3 \cdot S(P_3) = Q$
\end{center}

Отсюда определим быстроту действия насоса определим через значение потока:
\begin{center}
$S(P_3) = \frac{Q}{P_3}$
\end{center}



\subsection{Определение объёма вакуумной системы}
Процесс откачки описывается дифференциальным уравнением:
\[
\frac{dP}{dt} = -\frac{S(P) \cdot P}{V}
\]
где:
\begin{itemize}
    \item $P$ -- давление в системе, Торр
    \item $t$ -- время, с
    \item $S(P)$ -- быстрота действия насоса, л/с
    \item $V$ -- объём системы, л
\end{itemize}

При условии $S = const$ на выбранном участке, уравнение имеет решение:
\[
P(t) = P_0 \cdot \exp\left(-\frac{S}{V} \cdot t\right)
\]
где $P_0$ -- начальное давление при $t = 0$.

Прологарифмируем уравнение откачки:
\[
\ln P(t) = \ln P_0 - \frac{S}{V} \cdot t
\]

Получили линейную зависимость вида
\[
y = A + Bx
\]

Для расчётов был взят участок, на котором S почти линейно зависела от времени, найдено её среднее значение $S_{\text{ср}} = 33,8$ $\text{см}^{3}$/с, а также угловой коэффициент наклона зависимости $\ln P(t) = \ln P_0 - \frac{S}{V} \cdot t$, 
то есть отсюда $S/V \approx 0,00938 ~c^{-1}$.
Таким образом, $V \approx 3100$ $\text{см}^{3}$ или $0.0031 \text{м}^{3}$.



\begin{figure}[h!]
	\centering
	\includegraphics[width=21cm]{p9.PNG}
	\caption{ График зависимости быстродействия турбомолекулярного насоса от впуск
ного давления}
	\label{fig:Holl2}
\end{figure}

\begin{figure}[h!]
	\centering
	\includegraphics[width=21cm]{p6.PNG}
	\caption{Аппроксимация давления в высоковакуумной части}
	\label{fig:Holl2}
\end{figure}
\newpage

\subsection{Оценка применимости формулы}
\begin{enumerate}
    \item Быстрота действия насоса $S$ постоянна на выбранном временном интервале
    \item Отсутствуют значительные натекания газа в системе
    \item Температура системы постоянна
    \item Однотипный режим течения газа (молекулярный или вязкостный)
\end{enumerate}

\section{АНАЛИЗ РЕЗУЛЬТАТОВ}

\subsection*{Характеристики форвакуумного насоса}
\begin{itemize}
\item Быстрота действия форвакуумного насоса при атмосферном давлении составляет около 0,75 л/с
\item Зависимость $S(P)$ показывает относительную стабильность производительности насоса
\item Предельное давление в форвакуумной части, достигнутое в результате выполнения работы, составляет 0,016 Торр.
\end{itemize}

\subsection*{Характеристики турбомолекулярного насоса}
\begin{itemize}
\item Быстрота действия турбомолекулярного насоса варьируется от (8 до 36)$*10^{-3}$ $\text{м}^{3}$/с в рабочем диапазоне давлений
\item Насос работает в диапазоне давлений $10^{-5} - 10^{-3}$ Торр, предельное давление -- $1,7\cdot10^{-6}$ Торр
\item Зависимость $P_3(t)$ при несильно низких давлениях аппроксимируется функцией $P(t) = P_0 \cdot \exp\left(-\frac{S}{V} \cdot t\right)$
\end{itemize}

\subsection*{Анализ течения через диафрагму}
\begin{itemize}
\item Подтверждён молекулярный режим течения ($d \ll \lambda$)
\item Рассчитанная проводимость диафрагмы соответствует теоретическим ожиданиям
\item Поток через диафрагму стабилен в установившемся режиме
\end{itemize}

\section{ВЫВОДЫ}
В ходе проведённой работы были достигнуты следующие результаты:

Изучены и успешно применены на практике основы функционирования вакуумного оборудования, включая форвакуумный пластинчато-роторный, турбомолекулярный насосы и вакуумметры разных систем.

Проведён экспериментальный анализ рабочих характеристик используемых насосов.

Экспериментально подтверждена справедливость использования формулы молекулярного течения для расчёта газового потока через диафрагму.

Рассчитан теоретический объём вакуумной системы.

Достигнуто предельное остаточное давление в системе на уровне $10^{-6}$ Торр.

Применённые методики измерений в целом доказали свою эффективность, однако для некоторых задач были выявлены ограничения по их применению.
\section{Приложение}

\begin{figure}[h!]
	\centering
	\includegraphics[width=16cm]{p10.png}
	\caption{Зависимость давления от времени емкостного насоса В1}
	\label{fig:Holl2}
\end{figure}

\end{document}