\documentclass[a4paper]{article}
\usepackage[utf8]{inputenc}
\usepackage[russian,english]{babel}
\usepackage[T2A]{fontenc}
\usepackage{multirow}
\usepackage[left=10mm, top=20mm, right=18mm, bottom=15mm, footskip=10mm]{geometry}
\usepackage{indentfirst}
\usepackage{amsmath,amssymb}
\usepackage{booktabs}
\usepackage[italicdiff]{physics}
\usepackage{graphicx}
\graphicspath{{images/}}
\DeclareGraphicsExtensions{.pdf,.png,.jpg}
\usepackage{wrapfig}
\usepackage{subcaption}
\usepackage{caption}
\usepackage{array}
\usepackage{siunitx}
\usepackage{geometry}
\captionsetup[figure]{name=Рисунок}
\captionsetup[table]{name=Таблица}
  

\title{\underline{Отчет о выполненой лабораторной работе по электронике}}
\author{Воронин Денис, Б04-407}

\begin{document}

\maketitle

\begin{center}
\textbf{\Large Термоэлектронный диод}
\end{center}

\textbf{Цели работы:} практическое изучение явления термоэлектронной эмиссии и
процессов токопрохождения в вакууме, изготовление вакуумного
диода и исследование некоторых его характеристик. \par
\textbf{Оборудование:} самодельный диод из никилевой пластины и вольфрамовой нити, насос для откачки воздуха,гермитичная камера для проведения эксперимента (крепление диода), источник питания,
амперметр, вольтметр, соединительные провода.

\section{Теоретические сведения}

Термоэлектронной эмиссией (ТЭЭ) называется испускание
электронов поверхностью нагретых проводящих тел. \par
Явление термоэлектронной эмиссии и токопрохождение в вакууме
являются основными процессами, определяющими работу многих
электровакуумных приборов.

\begin{wrapfigure}[11]{r}{5.0cm}\vspace{-6mm}
  \centering
  \includegraphics[width=5cm]{p1.PNG}
  \caption{Конструкция простейшего термо-ого диода}
  \label{Theor}
\end{wrapfigure}

Простейший прибор для наблюдения ТЭЭ -- термоэлектронный диод, помещенный в объем с низким давлением остаточных газов, состоит из двух металлических электродов (рис. 1). Электрод, эмитирующий электроны, называется \emph{катодом} или \emph{эмиттером}. Получающий электроны электрод обычно
называется \emph{анодом} или \emph{коллектором}. При термоэмиссии катод нагрет до высокой температуры $T$, а анод имеет существенно более низкую температуру.
Прикладывая разность потенциалов между катодом и анодом $V_A$ и измеряя ток $I_A$, протекающий между электродами, получим вольтамперную характеристику (ВАХ) диода, т.е. зависимость $I_A = f(V_A)$.

\subsection{Формула Ричардсона-Дешмана}

Для прямоугольного потенциального барьера Ричардсон и Дэшман (1928) рассчитали максимальную плотность тока (тока насыщения) термоэлектронной эмиссии, которую может обеспечить при температуре $T$ термокатод с работой выхода электронов $\varphi$ (формула Ричардсона--Дэшмана):

\begin{equation}
j = A_0 (1 - \bar{r}) T^2 \exp \left( -\frac{\varphi}{kT} \right),
\end{equation}

где

\begin{equation}
A_0 = \frac{4\pi l_0 m k^2}{h^3} = 120.14 \, \text{A/см}^2\text{K}^2
\end{equation}

-- термоэмиссионная постоянная Зоммерфельда; $T$ -- температура катода по абсолютной шкале Кельвина; $\bar{r}$ -- коэффициент отражения электронов на границе металл-вакуум (обычно не превышает 0,07 и при оценочных расчетах им можно пренебречь); $\varphi$ -- 
работа выхода электронов из катода; $k = 1.38 \cdot 10^{-23} \, \text{Дж/К}$ -- постоянная Больцмана. Сила тока ТЭЭ определяется выражением $I = jS_k$, где $S_k$ -- площадь эмитирующей поверхности катода.

При больших напряжениях проявляется эффект Шоттки -- понижение потенциального барьера на границе металл-вакуум при приложении вытягивающего электроны электрического поля напряженностью $E$. Снижение работы выхода электронов, вызванное искажением барьера, определяется формулой

\begin{equation}
\Delta \varphi = e \sqrt{eE} \, (\text{в CГСЭ}) = 3,8 \cdot 10^{-4} \sqrt{E} [\text{эВ}],
\label{eq:3}
\end{equation}

где напряженность $E$ измеряется в В/см.

В формуле Ричардсона--Дэшмана учет эффекта Шоттки приводит лишь к умножению тока насыщения на $\exp(\Delta \varphi / kT)$. С учетом (\ref{eq:3}) выражение (1) для плотности тока насыщения принимает вид

\begin{equation}
j = A_0 (1 - \bar{r}) T^2 \exp\left(-\frac{\varphi - e \sqrt{eE}}{kT}\right).
\label{eq:4}
\end{equation}

Для плоскопараллельной геометрии диода и максвелловского распределения по скоростям электронов, вылетевших из катода, зависимость анодного тока определяется формулой

\begin{equation}
I_A = I_0 \exp\left(-\frac{eV_m}{kT}\right),
\label{eq:5}
\end{equation}

\subsection{Параметры термоэлектронных катодов}

Применяемые на практике термоэлектронные эмиттеры характеризуются эффективностью. Эффективность катода $H$ определяется отношением плотности тока насыщения к той удельной мощности накала $p_H$, которую необходимо подводить к единице поверхности катода для поддержания стационарных условий работы. Другими словами, эффективность -- это электронный ток, получаемый на единицу мощности накала $P_H$:

\begin{equation}
H = I / P_H = j / p_H.
\label{eq:14}
\end{equation}

Подставляя формулу Ричардсона--Дэшмана в (16) и пренебрегая степенной зависимостью $H$ от $T$, получим

\begin{equation}
H \approx C_1 \exp(-\varphi/kT).
\label{eq:15}
\end{equation}

Срок службы катода $\tau$ определяется в основном скоростью испарения рабочего вещества катода:

\begin{equation}
\tau = C_2 \exp(-q/kT),
\label{eq:16}
\end{equation}

где $q$ -- теплота испарения.

С повышением температуры эмиссионная способность и эффективность термокатода экспоненциально растут, а срок службы экспоненциально уменьшается. Обычно термокатоды имеют эффективность 5--100 мА/Вт, а срок службы -- от 5 до 100 тысяч часов. Критерием пригодности вещества для применения в качестве термокатода является условие $\varphi / q < 0,5$.

\subsection{Статические характеристики и параметры диода}

Крутизна диода в данной точке его характеристики определяется тангенсом наклона касательной к кривой $I_A = f(V_A)$ и приближенно равна отношению приращения тока анода к приращению на аноде:

Дифференцируем уравнение тока $I_A$ (формулу закона ``трёх вторых'') по $V_A$, получим аналитическое выражение крутизны в заданной точке:

\begin{equation}
S = \frac{dI_A}{dV_A} = \frac{3}{2} gV_A^{\frac{1}{2}}.
\end{equation}

В случае цилиндрической геометрии, как следует из (13), крутизна характеристики данного типа лампы определяется отношением $S_A / r_A^2$, т.е. она тем больше, чем длиннее катод и чем меньше расстояние катод-анод.

Внутреннее сопротивление -- величина, обратная крутизне. От этого сопротивления необходимо отличать сопротивление лампы постоянному току, которое определяется по закону Ома и равно:

\begin{equation}
R_0 = \frac{V_A}{I_A} = \frac{V_A}{gV_A^{3/2}} = \frac{1}{gV_A^{1/2}}.
\label{eq:18}
\end{equation}

Пересанс определяется как отношение тока к анодному напряжению в степени трех вторых (12). Пересанс не зависит от анодного напряжения и наиболее полно характеризует электронный поток.

Обратное напряжение $V_{\text{обр}}$ характеризует максимальное допустимое напряжение между анодом и катодом, которое выдерживает лампа без пробоя при подаче на анод отрицательного напряжения.

Допустимая мощность. Электроны, падающие на анод, передают свою кинетическую энергию. Температура анода вследствие этого повышается. Энергия, сообщенная аноду электронами на единицу времени, определяется выражением $P_A = I_A V_A$. Кроме того, на анод поступает часть тепла, излучаемого катодом. Полную мощность, выделяемую на аноде, следует подсчитывать по формуле

\begin{equation}
P_A = I_A V_A + \alpha P_H, \quad \text{где } \alpha = 0,2 - 1,0.
\label{eq:19}
\end{equation}

\newpage

\section{Схема и описание экспериментальной установки }

Схема экспериментальной установки выгляди следующим образом:

\begin{figure}[h!]
	\centering
	\includegraphics[width=8cm]{p2.PNG}
	\caption{Электрическая схема для измерений характеристик диода}
	\label{fig:Holl2}
\end{figure}

где, 1 -- исследуемый диод, 2 -- регулируемый источник тока накала, 3 -- вольтметр, 4 -- регулируемый источник напряжения, 5 -- амперметр.

\section{Порядок выполнениея работы}

\subsection{Изготовление диода}

\begin{wrapfigure}[11]{l}{5.0cm}\vspace{-6mm}
  \centering
  \includegraphics[width=4cm]{p3.PNG}
  \caption{Вид диода в сборке}
  \label{Theor}
\end{wrapfigure}

Сначала свернули никелевую пластину в цилиндр, затем при помощи сварки скрепили края цилиндра вместе.\par
Затем сложили никилевую проволоку по нужной геометрии и припаяли сбоку цилиндра (отвод анода)\par
Следующим шагом припаяли 'анодную часть' к креплению, с помощью которого диод будет крепиться в установке.\par
Далее делаем катодную часть с помошью никелевой проволоки и вольфрамовой нити и припаеваем к основной конструкции.\par
После этого проверяем работоспособность диода и приступаем к измерениям.

\begin{figure}[h!]
	\centering
	\includegraphics[width=7cm]{p4.jpg}
	\caption{Диод в сборке}
	\label{fig:Holl2}
\end{figure}

\newpage

\subsection{Проведение измерений}\par

Перед началом измерим толщину никелевой пластины (0,2 мм) и диаметр вольфрамовой проволоки (0,05 мм )\par

Перед началом измерений давление в камере было $6*10^{-5}$ торр.


\begin{table}[h!]
\centering
\setlength{\tabcolsep}{2.5pt}
\caption{Прогрев катода}
\begin{tabular}{c|cccccccccccccccccccc}
\toprule
\( I_\text{знак} \), А & 0,01 & 0,02 & 0,03 & 0,04 & 0,05 & 0,06 & 0,07 & 0,08& 0,09 & 0,1 & 0,2 & 0,3 & 0,4 & 0,5 & 0,6 & 0,7 & 0,8 & 0,9 & 1,00 \\
\midrule
\( U_\text{знак} \), мВ & - & - & - & - & - & - & - & - & - & 21 & 43 & 66 & 96 & 132 & 178 & 239 & 317 & 404 & 514 \\
\bottomrule
\end{tabular}
\end{table}

\begin{table}[h!]
\centering
\setlength{\tabcolsep}{2.5pt}
\begin{tabular}{c|cccccccccccccccccccc}
\toprule
\( I_\text{знак} \), А  & 1,1 & 1,2 & 1,3 & 1,4 & 1,5 & 1,6 & 1,7 & 1,8 & 1,9 & 2 & 2,1 & 2,2 & 2,3 & 2,4 & 2,5 & 2,6 & 2,7 & 2,8 & 2,9 \\
\midrule
\( U_\text{знак} \), В & 0,729 & 0,906 & 1,041 & 1,209 & 1,452 & 1,578 & 1,852 & 1,991 & 2,27 & 2,48 & 2,74 & 2,97 & 3,14 & 3,41 & 3,73 & 4,12 & 4,34 & 4,66 & 5,09 \\
\bottomrule
\end{tabular}
\end{table}

Давление в камере после прогрева $1*10^{-4}$ Торр \par






Измерение вольт-амперных характеристик диода 

\begin{table}[h!]
\centering
\caption{\( I_{\text{нак}} = 2{,}4\,\text{A},\quad U_{\text{нак}} = 3{,}41\,\text{B} \)}
\setlength{\tabcolsep}{2.5pt}
\begin{tabular}{c|cccccccccccccccccccc}
\toprule
\( U_a \), В & 1 & 2 & 3 & 4 & 5 & 10 & 20 & 30 & 40 & 50 & 60 & 70 & 80 & 90 & 100 & 110 & 120 & 130 & 140 \\
\midrule
\( I_a \), мкА & 17,00 & 36,00 & 46,00 & 47,00 & 48,00 & 51,00 & 53,00 & 55,00 & - & - & - & - & - & - & - & - & - & - & - \\
\midrule
\( \sigma I_a \) & 0,03 & 0,03 & 0,03 & 0,03 & 0,03 & 0,03 & 0,03 & 0,03 & - & - & - & - & - & - & - & - & - & - & - \\
\bottomrule
\end{tabular}
\end{table}


\begin{table}[h!]
\centering
\caption{\( I_{\text{нак}} = 2{,}5\,\text{A},\quad U_{\text{нак}} = 3{,}73\,\text{B} \)}
\setlength{\tabcolsep}{2.5pt}
\begin{tabular}{c|cccccccccccccccccccc}
\toprule
\( U_a \), В & 1 & 2 & 3 & 4 & 5 & 10 & 20 & 30 & 40 & 50 & 60 & 70 & 80 & 90 & 100 & 110 & 120 & 130 & 140 \\
\midrule
\( I_a \), мкА & 23,00 & 52,00 & 79,00 & 82,00 & 83,00 & 86,00 & 90,00 & 92,00 & - & - & - & - & - & - & - & - & - & - & - \\
\midrule
\( \sigma I_a \) & 0,03 & 0,03 & 0,03 & 0,03 & 0,03 & 0,03 & 0,03 & 0,03 & - & - & - & - & - & - & - & - & - & - & - \\
\bottomrule
\end{tabular}
\end{table}

\begin{table}[h!]
\centering
\caption{\( I_{\text{нак}} = 2{,}6\,\text{A},\quad U_{\text{нак}} = 4{,}12\,\text{B} \)}
\setlength{\tabcolsep}{2.5pt}
\begin{tabular}{c|cccccccccccccccccccc}
\toprule
\( U_a \), В & 1 & 2 & 3 & 4 & 5 & 10 & 20 & 30 & 40 & 50 & 60 & 70 & 80 & 90 & 100 & 110 & 120 & 130 & 140 \\
\midrule
\( I_a \), мкА & 43,00 & 111,00 & 203,00 & 280,00 & 334,00 & 350,00 & 367,00 & 378,00 & — & — & — & — & — & — & — & — & — & — & — \\
\midrule
\( \sigma I_a \) & 0,03 & 0,03 & 0,03 & 0,03 & 0,03 & 0,03 & 0,03 & 0,03 & - & - & - & - & - & - & - & - & - & - & - \\
\bottomrule
\end{tabular}
\end{table}

\begin{table}[h!]
\centering
\caption{\( I_{\text{нак}} = 2{,}7\,\text{A},\quad U_{\text{нак}} = 4{,}34\,\text{B} \)}
\setlength{\tabcolsep}{1.5pt}
\begin{tabular}{c|cccccccccccccccccccc}
\toprule
\( U_a \), В & 1 & 2 & 3 & 4 & 5 & 10 & 20 & 30 & 40 & 50 & 60 & 70 & 80 & 90 & 100 & 110 & 120 & 130 & 140 \\
\midrule
\( I_a \), мкА & 70,00 & 164,00 & 264,00 & 437,00 & 595,00 & 1273,00 & 1356,00 & 1402,00 & 1455,00 & 1617,00 & - & 1700,00 & - & 1750 & 1780 & - & - & 1860 & - \\
\midrule
\( \sigma I_a \) & 0,03 & 0,03 & 0,03 & 0,03 & 0,03 & 0,03 & 0,03 & 0,03 & 0,03 & 0,03 & - & 0,03 & - & 0,03 & 0,03 & - & - & 0,03 & - \\
\bottomrule
\end{tabular}
\end{table}

\begin{table}[h!]
\centering
\caption{\( I_{\text{нак}} = 2{,}8\,\text{A},\quad U_{\text{нак}} = 4{,}66\,\text{B} \)}
\setlength{\tabcolsep}{1.5pt}
\begin{tabular}{c|cccccccccccccccccccc}
\toprule
\( U_a \), В & 1 & 2 & 3 & 4 & 5 & 10 & 20 & 26 & 40 & 50 & 60 & 70 & 80 & 90 & 100 & 110 & 120 & 130 & 140 \\
\midrule
\( I_a \), мкА & 75,00 & 178 & 313,00 & 475,00 & 660,00 & 1970,00 & 3910,00 & 4070,00 & — & 4740,00 & — & 4930,00 & - & 5230 & 5360 & 5550 & 5660 & 6170 & 6210 \\
\midrule
\( \sigma I_a \) & 0,03 & 0,03 & 0,03 & 0,03 & 0,03 & 0,03 & 0,05 & 0,05 & - & 0,05 & - & 0,05 & - & 0,05 & 0,05 & 0,05 & 0,05 & 0,05 & 0,05 \\
\bottomrule
\end{tabular}
\end{table}

\begin{table}[h!]
\centering
\setlength{\tabcolsep}{1.0pt}
\caption{\( I_{\text{нак}} = 2{,}9\,\text{A},\quad U_{\text{нак}} = 5{,}09\,\text{B} \)}
\begin{tabular}{c|cccccccccccccccccccc}
\toprule
\( U_a \), В & 1 & 2 & 3 & 4 & 5 & 10 & 20 & 26 & 40 & 50 & 60 & 70 & 80 & 90 & 100 & 110 & 120 & 130 & 140 \\
\midrule
\( I_a \), мкА & 90,00 & 200,00 & 330,00 & 510,00 & 720,00 & 2150,00 & 6170,00 & 9010,00 & 13700 & 14400 & 14650 & 14790 & 14860 & 14900 & 14900 & — & — & 15000 & — \\
\midrule
\( \sigma I_a \) & 0,03 & 0,03 & 0,03 & 0,03 & 0,03 & 0,03 & 0,05 & 0,05 & 0,05 & 0,05 & 0,05 & 0,05 & 0,05 & 0,05 & 0,05 & - & - & 0,05 & - \\
\bottomrule
\end{tabular}
\end{table}


\newpage
\subsection{Построение графиков}

\begin{figure}[h!]
	\centering
	\includegraphics[width=18cm]{g1.png}
	\caption{График зависимости тока накала от напряжения накала.}
	\label{fig:Holl2}
\end{figure}


\begin{figure}[h!]
	\centering
	\includegraphics[width=18cm]{g2.png}
	\caption{График зависимости сопротивления катода от
приложенной мощности.}
	\label{fig:Holl2}
\end{figure}
\newpage

\begin{figure}[h!]
	\centering
	\includegraphics[width=16.5cm]{g3.png}
	\caption{График зависимости температуры катода от тока
накала через 1 - изменения
сопротивления, 2 - на основании расчетов с
использованием уравнения энергетического
баланса, 3 - на основании расчётов с
использованием уравнения Ричардсона–
Дэшмана.}
	\label{fig:Holl2}
\end{figure}

\begin{figure}[h!]
	\centering
	\includegraphics[width=16.5cm]{g4.png}
	\caption{График зависимости анодного тока от анодного
напряжения при различных значениях тока накала IH в
координатах lg(IA) от lg(VA).}
	\label{fig:Holl2}
\end{figure}


\begin{figure}[h!]
	\centering
	\includegraphics[width=18cm]{g5.png}
	\caption{График зависимости анодного тока от анодного
напряжения при различных значениях тока накала IH в
координатах lg(IA) от lg(VA)(С апроксимацией)}
	\label{fig:Holl2}
\end{figure}


\begin{figure}[h!]
	\centering
	\includegraphics[width=18cm]{g6.png}
	\caption{Расчет КПД вакуумного диода}
	\label{fig:Holl2}
\end{figure}

\begin{figure}[h!]
	\centering
	\includegraphics[width=18cm]{g7.png}
	\caption{График зависимости анодного тока от тока накала}
	\label{fig:Holl2}
\end{figure}



\newpage




\begin{table}[htbp]
\centering
\caption{Экспериментальные данные(рис 6)}
\label{tab:exp_data}
\begin{tabular}{cccc}
\toprule
I (A) & U (V) & P (W) & R ($\Omega$) \\
\midrule
0.1 & 0.021 & 0.002 & 0.210 \\
0.2 & 0.043 & 0.009 & 0.215 \\
0.3 & 0.066 & 0.020 & 0.220 \\
0.4 & 0.096 & 0.038 & 0.240 \\
0.5 & 0.132 & 0.066 & 0.264 \\
0.6 & 0.178 & 0.107 & 0.297 \\
0.7 & 0.239 & 0.167 & 0.341 \\
0.8 & 0.317 & 0.254 & 0.396 \\
0.9 & 0.404 & 0.364 & 0.449 \\
1.0 & 0.514 & 0.514 & 0.514 \\
1.1 & 0.729 & 0.802 & 0.663 \\
1.2 & 0.906 & 1.087 & 0.755 \\
1.3 & 1.041 & 1.353 & 0.801 \\
1.4 & 1.209 & 1.693 & 0.864 \\
1.5 & 1.452 & 2.178 & 0.968 \\
1.6 & 1.578 & 2.525 & 0.986 \\
1.7 & 1.852 & 3.148 & 1.089 \\
1.8 & 1.991 & 3.584 & 1.106 \\
1.9 & 2.270 & 4.313 & 1.195 \\
2.0 & 2.480 & 4.960 & 1.240 \\
2.1 & 2.740 & 5.754 & 1.305 \\
2.2 & 2.970 & 6.534 & 1.350 \\
2.3 & 3.140 & 7.222 & 1.365 \\
2.4 & 3.410 & 8.184 & 1.421 \\
2.5 & 3.730 & 9.325 & 1.492 \\
2.6 & 4.120 & 10.712 & 1.585 \\
2.7 & 4.340 & 11.718 & 1.607 \\
2.8 & 4.660 & 13.048 & 1.664 \\
2.9 & 5.090 & 14.761 & 1.755 \\
\bottomrule
\end{tabular}
\end{table}


\section{К рисунку 7}

\begin{enumerate}
    \item \textbf{Первый график:}
    \[
    T = T_0 + \frac{\left( \dfrac{R_T}{R_0} - 1 \right)}{\alpha},
    \]
    где 
    \[
    T_0 = \SI{273}{\kelvin}, \quad
    R_0 = \SI{0.145}{\ohm}, \quad
    \alpha = \SI{9.29e-3}{\per\kelvin}, \quad
    R_T = \frac{U}{I}.
    \]

    \item \textbf{Второй график:}
    \[
    T = \left( \frac{P_{\text{накала}}}{\varepsilon \sigma S_{\text{изл}}} \right)^{1/4},
    \]
    где 
    \[
    P_{\text{накала}} = \frac{U}{I}, \quad
    \varepsilon = 0.25, \quad
    \sigma = \SI{5.67e-12}{\watt\per\centi\meter\squared\per\kelvin\tothe{4}}, \quad
    S_{\text{изл}} = \SI{0.1884}{\centi\meter\squared}.
    \]

    \item \textbf{Третий график:}
    Используется формула Ричардсона — Дешмана для плотности насыщенного тока термоэлектронной эмиссии:
    \[
    j_{\text{нас}} = A_0 (1 - \tau) T^2 \exp\!\left( -\frac{\varphi}{k T} \right),
    \]
    где 
    \[
    \varphi \approx \SI{4.5}{\electronvolt}, \quad
    A_0 = \SI{120.4}{\ampere\per\centi\meter\squared\per\kelvin\squared}, \quad
    j_{\text{нас}} = \frac{I_{\text{нас}}}{S_{\text{к}}}.
    \]
\end{enumerate}

\begin{table}[h]
\centering
\caption{Зависимость $g$ от $I_{\text{пак}}$}
\label{tab:current_g}
\begin{tabular}{c|c}
$I_{\text{пак}}$ [A] & $g$ [мкА/В$^{3/2}$] \\
\hline
2.4 & 5.65 \\
2.5 & 9.62 \\
2.6 & 32.83 \\
2.7 & 53.57 \\
2.8 & 59.48 \\
2.9 & 64.45 \\
\end{tabular}
\end{table}


\section{К рисунку 8 и 9}
\subsection*{Экспериментальные значения $g$}
\begin{tabular}{c|c}
$I_{\text{нак}}$ [A] & $g$ [мкА/В$^{3/2}$] \\
\hline
2.4 & 5,65 \\
2.5 & 9,62 \\
2.6 & 32,83 \\
2.7 & 53,57 \\
2.8 & 59,58 \\
2.9 & 64,45 \\
\end{tabular}

\vspace{1em}


\subsection*{Теоретическое значение $g$}
\[
g_{\text{теор}} = 104.72~\text{мкА/В}^{3/2}
\]


\vspace{1em}


\subsection*{Сравнение экспериментальных и теоретических значений}
\begin{tabular}{c|c|c}
$I_{\text{нак}}$ [A] & $g_{\text{эксп}}$ [мкА/В$^{3/2}$] & Отношение $g_{\text{эксп}} / g_{\text{теор}}$ \\
\hline
2.4 & 5,65 & 0.165 \\
2.5 & 9,62 & 0.174 \\
2.6 & 32,83 & 0.347 \\
2.7 & 53,58 & 0.622 \\
2.8 & 59,48 & 0.643 \\
2.9 & 64,45 & 0.691 \\
\end{tabular}

\vspace{1em}


\subsection*{Анализ зависимости $g$ от тока накала}
С увеличением тока накала $I_{\text{нак}}$:

Увеличивается эмиссия электронов с катода\par
Увеличивается коэффициент $g$\par
Это соответствует увеличению крутизны ВАХ\par

\section{К рисунку 9}

\section*{ВЫЧИСЛЕНИЕ ОТНОШЕНИЯ $e/m$}

\begin{align*}
d &= \SI{1.50e-4}{\meter} \\
S &= \SI{1.88e-5}{\meter\squared} \\
g &= \SI{1.01e-4}{\ampere\per\volt\tothe{3/2}} \\
\varepsilon_0 &= \SI{8.85e-12}{\farad\per\meter}
\end{align*}

\subsection*{Результат:}
\begin{align*}
e/m &= \SI{4.73e10}{\text{КЛ}\per\kilogram} \\
e/m &= \SI{4.73e10}{\text{КЛ}\per\kilogram} 
\end{align*}

\subsection*{Сравнение с табличным значением:}
\begin{align*}
\text{Табличное значение: } & \SI{1.76e11}{\text{КЛ}\per\kilogram} \\
\text{Отклонение: } & 73.1\%
\end{align*}


\section{К рисунку 10}
\section*{МАКСИМАЛЬНЫЕ ЗНАЧЕНИЯ КПД ДЛЯ КАЖДОГО ТОКА НАКАЛА}

\begin{table}[h]
\centering
\begin{tabular}{ccccccc}
\toprule
$I_{\text{нак}}$ [A] & $U_{\text{нак}}$ [B] & $P_{\text{нак}}$ [Br] & $U_a$ [B] & $I_a$ [мкА] & $P_a$ [мкВт] & КПД [\%] \\
\midrule
2.4 & 3.41 & 8.184 & 30 & 55 & 1650.0 & 0.020 \\
2.5 & 3.73 & 9.325 & 30 & 92 & 2760.0 & 0.030 \\
2.6 & 4.12 & 10.712 & 30 & 378 & 11340.0 & 0.106 \\
2.6 & 4.12 & 10.712 & 30 & 378 & 11340.0 & 0.106 \\
2.7 & 4.34 & 11.718 & 130 & 1860 & 241800.0 & 2.022 \\
2.8 & 4.66 & 13.048 & 140 & 6210 & 869400.0 & 6.247 \\
2.9 & 5.09 & 14.761 & 130 & 15000 & 1950000.0 & 11.669 \\
\bottomrule
\end{tabular}
\end{table}

\section*{АНАЛИЗ РЕЗУЛЬТАТОВ}

\begin{enumerate}
\item КПД вакуумного диода очень низкий (менее 0.1\%)
\item С увеличением тока накала максимальный КПД увеличивается
\item Основная потребляемая мощность - мощность накала
\item Вакуумные диоды неэффективны как преобразователи энергии
\item Основное применение вакуумных диодов - выпрямление и детектирование, а не энергопреобразование
\end{enumerate}


\section{К рисунку 11}
\section*{ЗАВИСИМОСТЬ АНОДНОГО ТОКА ОТ ТОКА НАКАЛА}

\begin{table}[h]
\centering
\small
\begin{tabular}{c|cccccc}
\toprule
\multirow{2}{*}{$U_a$ [B]} & \multicolumn{6}{c}{$I_a$ [мкА] (lg($I_a$))} \\
\cmidrule{2-7}
 & 2.4 A & 2.5 A & 2.6 A & 2.7 A & 2.8 A & 2.9 A \\
\midrule
1   & 17 (1.23)  & 23 (1.36)  & 43 (1.63)  & 70 (1.85)  & 75 (1.88)  & 90 (1.95)  \\
5   & 48 (1.68)  & 83 (1.92)  & 334 (2.52) & 595 (2.77) & 660 (2.82) & 720 (2.86) \\
10  & 51 (1.71)  & 86 (1.93)  & 350 (2.54) & 1273 (3.10) & 1970 (3.29) & 2150 (3.33) \\
20  & 53 (1.72)  & 90 (1.95)  & 367 (2.56) & 1356 (3.13) & 3010 (3.59) & 6170 (3.79) \\
30  & 55 (1.74)  & 92 (1.96)  & 378 (2.58) & 1402 (3.15) & - & 9010 (3.95) \\
50  & - & - & - & 1617 (3.21) & 4740 (3.68) & 14400 (4.16) \\
100 & - & - & - & 1780 (3.25) & 5360 (3.73) & 14900 (4.17) \\
\bottomrule
\end{tabular}
\end{table}

\section*{АНАЛИЗ РЕЗУЛЬТАТОВ}

\begin{enumerate}
\item С увеличением тока накала анодный ток увеличивается
\item Зависимость $\lg(I_a)$ от $I_a$ близка к линейной
\item Чем выше анодное напряжение, тем больше анодный ток при том же токе накала
\item При малых $U_a$ (1-5 В) зависимость более слабая
\item При больших $U_a$ (20-100 В) наблюдается более сильная зависимость
\item Это объясняется увеличением термоэлектронной эмиссии с ростом температуры катода
\end{enumerate}

\section{Вывод}

Изучил явления термоэлектронной эмиссии и процесов токопрохождения в вакууме, изготовил вакуумный диод 
и иследовал некоторые его характеристики. Выяснил зависимость и построил график тока накала от напряжения (зависимость напоминает график корня при небольших напряжениях и токах). Также выяснил, как
зависит напряжение катода от приложенной к нему мощности и по имеющимся данным построил график. Была выяснена
зависимость температуры накала от тока накала с помощью 3 методов: метод определения по сопротивлению показал наибольшее отклонение от остальных 2-х. Возможно, это связано с неточностью взятия геометрических параметров диода, а также с несовсем корректно рабочим источником питания и напряжения.
Далее было посчитанно теоретическое и экспериментальное значение первеанса диода g, была выяснена его зависимость: чем больше ток накакла, тем меньше отклонение от теоретического значения.Возможно, если бы мы сделали еще пару значений тока накакла, то удалось бы минимизировать эту разницу.
Было вычисленно отношение e/m: оно отличилочь от табличного более чем в 3 раза, это связано с несовершенством снятых нами характеристик.
Были определены значения КПД для кажого тока накала и зависимость аноного тока накала. Их результаты описаны выше. 

\end{document}