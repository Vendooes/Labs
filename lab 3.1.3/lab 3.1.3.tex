\documentclass[a4paper]{article}
\usepackage[utf8]{inputenc}
\usepackage[russian,english]{babel}
\usepackage[T2A]{fontenc}
\usepackage[left=10mm, top=20mm, right=18mm, bottom=15mm, footskip=10mm]{geometry}
\usepackage{indentfirst}
\usepackage{amsmath,amssymb}
\usepackage[italicdiff]{physics}
\usepackage{graphicx}
\graphicspath{{images/}}
\DeclareGraphicsExtensions{.pdf,.png,.jpg}
\usepackage{wrapfig}
\usepackage{subcaption}
\usepackage{caption}
\captionsetup[figure]{name=Рисунок}
\captionsetup[table]{name=Таблица}
  
\title{\underline{Отчет о выполненой лабораторной работе 3.1.3}}
\author{Воронин Денис, Б04-407}

\begin{document}

\maketitle

\begin{center}
\textbf{\Large Измерение магнитного поля Земли}
\end{center}

\textbf{Цель работы:} определить характеристики шарообразных неодимовых магнитов и, используя законы взаимодействия магнитных моментов с полем, измерить горизонтальную и вертикальную составляющие индукции магнитного поля Земли и магнитное наклонение.

\textbf{В работе используются:} 12 одинаковых неодимовых магнитных шариков, тонкая нить для изготовления крутильного маятника, медная проволока диаметром \( (0,5 - 0,6) \, \text{мм} \), электронные весы, секундомер, измеритель магнитной индукции АТЕ-8702, штангенциркуль,
брусок из немагнитного материала \( (25 \times 30 \times 60 \, \text{мм}^3) \), деревянная линейка, штатив из немагнитного материала; дополнительные неодимовые магнитные шарики (\( \approx 20 \, \text{шт.} \)) и неодимовые магниты в форме параллелепипедов \( (2 \, \text{шт.} \)), набор гирь и разновесов.

\section{\text{Теоретические сведения}}

Простейший магнитный диполь может быть образован витком с током или постоянным магнитом. По определению, магнитный момент $\vec{P}_m$ тонкого витка площадью $S$ с током $I$ равен:

\[
\vec{P}_m = (I/c) \vec{S} = (I/c) S \vec{n}
\]

где $c$ — скорость света в вакууме, $\vec{S} = S \vec{n}$ — вектор площади контура, образующий с направлением тока правовинтовую систему, $\vec{n}$ — единичный вектор нормали к площадке $S$ (это же направление $\vec{P}_m$ принимается за направление $S \rightarrow N$ от южного ($S$) к северному ($N$) полюсу). Если размеры контура с током или магнитной стрелки малы по сравнению расстоянием до диполя, то соответствующий магнитный диполь $\vec{P}_m$ называют элементарным или точечным.

Магнитное поле точечного диполя определяется по формуле, аналогичной формуле для поля элементарного электрического диполя:

\[\overrightarrow{B} = \frac{3(\overrightarrow{P_{m}}\overrightarrow{ r})\overrightarrow{r}}{r^{5}}- \frac{\overrightarrow{P_{m}} }{r^{3}}\] 


В магнитном поле с индукцией $\vec{B}$ на точечный магнитный диполь $\vec{P}_m$ действует механический момент сил:

\[
\vec{M} = \vec{P}_m \times \vec{B}.
\]

Под действием вращающего момента $\vec{M}$ виток с током или постоянный магнит поворачивается так, чтобы его магнитный момент выстроился вдоль вектора индукции магнитного поля. Это — положение устойчивого равновесия: при отклонении от этого положения возникает механический момент внешних сил, возвращающий диполь к положению равновесия. В положении, когда $\vec{P}_m$ и $\vec{B}$ параллельны, но направлены противоположно друг другу, также имеет место равновесие ($M = 0$), но такое равновесие неустойчиво: малыйское отклонение от этого положения приведёт к появлению момента сил, стремящихся отклонить диполь ещё дальше от начального положения.

Магнитный диполь в магнитном поле обладает энергией:

\[
W = -(\vec{P}_m, \vec{B}).
\]
\par
Энэргия минимальна, когда сонаправлены векторы $\overrightarrow{ P_{m}}\upuparrows\overrightarrow{  B}$
\par

В неоднородном поле на точечный магнитный диполь, кроме момента сил, действует ещё и сила:
\[
\vec{F} = (\vec{P}_m, \vec{\nabla}) \vec{B},
\]

Последняя формула аналогична формуле для силы, действующей на электрический диполь в электрическом поле:
\[
\vec{F} = (\vec{P}_m, \vec{\nabla}) \vec{E}.
\]

Используя формулы для момента силы, силы и энергии, нетрудно выяснить, как ведёт себя свободный магнитный диполь в неоднородном магнитном поле: он выстраивается вдоль силовых линий магнитного поля и, кроме того, под действием результирующей силы, возникающей из-за неоднородности поля, втягивается в область более сильного магнитного поля, т.е. в область, где он обладает меньшей энергией.

Зная магнитные моменты $P_1$ и $P_2$ двух небольших постоянных магнитов, можно рассчитать силу их взаимодействия. Если магнитные моменты
\[
P_1 = P_2 = P_m
\]
двух одинаковых небольших магнитов направлены вдоль соединяющей их прямой, а расстояние между ними равно $r$, то магниты взаимодействуют с силой:
\[
F = P_m \frac{\partial B}{\partial r} = P_m \frac{\partial (2P_m/r^3)}{\partial r} = -\frac{6P_m^2}{r^4}.
\]

Магниты притягиваются, если их магнитные моменты направлены одинаково $(\vec{P}_1 \uparrow\uparrow \vec{P}_2)$ и отталкиваются, если моменты направлены противоположно друг другу $(\vec{P}_1 \uparrow\downarrow \vec{P}_2)$.

Если магнитные моменты направлены перпендикулярно соединяющей их прямой, то сила их взаимодействия окажется в два раза меньшей:
\[
F = \frac{3P_m^2}{r^4};
\]
в этом случае диполи притягиваются при $\vec{P}_1 \uparrow\uparrow \vec{P}_2$ и отталкиваются при $\vec{P}_1 \uparrow\downarrow \vec{P}_2$.

Полный магнитный момент $\vec{P}_m$ постоянного магнита определяется намагниченностью $\vec{J}$ вещества, из которого он изготовлен. По определению, намагниченность – это магнитный момент единицы объёма. Для однородно намагниченного шара намагниченность, очевидно, равна:

\[
\vec{p_{m}} = \vec{P}_m / V,
\]

где $V$ — объём шара.\par
Остаточная магнитная индукция определяется по формуле:
\[B_{r} = 4\pi p_{m}\]

Индукция магнитного поля $B_s$ на полюсах однородно намагниченного шара связана с величиной намагниченности $\vec{p_{m}}$ и остаточной магнитной индукцией $B_r$ формулой:

\[
\vec{B_p} = (8\pi/3)\vec{p_{m}} = (2/3)\vec{B_r} (2)
\]

\subsection{Определения магнитного момента}

\begin{wrapfigure}{l}{0.4\textwidth}
    \centering
    \includegraphics[width=0.38\textwidth]{p1.png}
    \caption{Определение магнитного момента шариков по силе тяжести}
    \label{fig:example}
\end{wrapfigure}

Величину магнитного момента $P_m$ одинаковых шариков можно рассчитать, зная их массу $m$ и определив максимальное расстояние $r_{\text{max}}$, на котором они ещё удерживают друг друга в поле тяжести (см. рис. 1). При максимальном расстоянии сила тяжести шариков равна силе их магнитного притяжения:

\[
\frac{6P_m^2}{r_{\text{max}}^4} = mg, \quad P_m = \sqrt{\frac{mgr_{\text{max}}^4}{6}}. (1)
\]

По величине магнитного момента $P_m$ можно рассчитать величину индукции магнитного поля вблизи любой точки на поверхности шара радиуса $R$. Максимальная величина индукции наблюдается на полюсах:

\[
\vec{B_p} = \frac{2\vec{P_m}}{R^3}.
\]

\section{Экспериментальная установка}

\subsection{Измерение горизонтальной составляющей индукции магнитного поля}

\begin{wrapfigure}{l}{0.4\textwidth}
    \centering
    \includegraphics[width=0.32\textwidth]{p2.png}
    \caption{Крутильный маятник}
    \label{fig:example}
\end{wrapfigure}

Магнитное поле Земли в настоящей работе определяется по периоду крутильных колебаний магнитной стрелки вокруг вертикальной оси.

«Магнитная стрелка» образована из сцепленных друг с другом противоположными полюсами шариков и с помощью $\Lambda$-образного подвеса подвешена в горизонтальном положении (см. рис. 3). Магнитные моменты шариков направлены в одну сторону вдоль оси «стрелки». Под действием вращательного момента $\vec{M} = \vec{P}_0 \times \vec{B}$ магнитный момент «стрелки» $\vec{P}_0$ выстроится вдоль горизонтальной составляющей магнитного поля Земли $\vec{B}_h$ в направлении Юг $\rightarrow$ Север. При отклонении «стрелки» на угол $\theta$ от равновесного положения в горизонтальной плоскости возникают крутильные колебания вокруг вертикальной оси, проходящей через середину стрелки. Если пренебречь упругостью нити, то уравнение крутильных колебаний такого маятника определяется возвращающим моментом сил $M = -P_0 B_h \sin \theta$, действующим на «стрелку» со стороны магнитного поля Земли, и моментом инерции $I_n$ «стрелки» относительно оси вращения.

При малых амплитудах ($\sin \theta \approx \theta$) уравнение колебаний «стрелки» имеет вид:

\[
I_n \frac{d^2 \theta}{dt^2} = -P_0 B_h \theta, \quad \text{или} \quad I_n \ddot{\theta} + P_0 B_h \theta = 0.
\]

Период колебаний

\[
T = 2\pi \sqrt{\frac{I_n}{P_0 B_h}} = 2\pi \sqrt{\frac{I_n}{n P_m B_h}},
\]

где $P_0 = n P_m$ — полный магнитный момент магнитной «стрелки», составленной из $n$ шариков.

---
В нашем приближении период колебаний маятника оказывается пропорциональным квадратному корню из числа пар $n$, составляющих «стрелку»:

\[
T(n) = 2\pi\sqrt{\frac{I_n}{n P_m B_h}} = 2\pi\sqrt{\frac{n m d^2/12}{n P_m B_h}} = \pi n\sqrt{\frac{md^{2}}{3 P_m B_h}} = kn,
\]

где $k = \pi d\sqrt{\frac{m}{3 P_m B_h}}$.

\subsection{Измерение вертикальной составляющей 
индукции магнитного поля Земли.  
Магнитное наклонение.}

\begin{wrapfigure}{l}{0.4\textwidth}
    \centering
    \includegraphics[width=0.27\textwidth]{p3.png}
    \caption{Определение вертикальной составляющей поля Земли}
    \label{fig:example}
\end{wrapfigure}

Для измерения вертикальной $B_v$ составляющей вектора индукции поля Земли используется та же установка, что и для измерения горизонтальной составляющей с тем лишь отличием, что магнитная «стрелка» подвешивается на нити без $\Lambda$-образного подвеса. В этом случае магнитная «стрелка», составленная из чётного числа шариков и подвешенная на тонкой нити за середину, расположится не горизонтально, а под некоторым, отличным от нуля, углом к горизонту.

С помощью небольшого дополнительного грузика «стрелку» можно «выровнять», расположив её горизонтально: в этом случае момент силы тяжести груза относительно точки подвеса будет равен моменту сил, действующих на «стрелку» со стороны магнитного поля Земли. Если масса уравновешивающего груза равна $m_{\text{гр}}$, плечо силы тяжести $r_{\text{гр}}$, а полный магнитный момент «стрелки» $P_0 = nP_m$, то в равновесии:

\[
m_{\text{гр}} g r_{\text{гр}} = P_0 B_v = n P_m B_v
\]

($B_v$ — вертикальная составляющая поля Земли). Видно, что момент $M(n)$ силы тяжести уравновешивающего груза пропорционален числу $n$ шариков, образующих магнитную «стрелку»:

\[
M(n) = A n, \quad \text{где} \quad A = P_m B_v.\]

\section{Ход работы}

\subsection{Задание 1}

Измерили массу 19 шариков $M = 15,746\pm 0,005 \text{г}$ и потом нашли массу одного $m = 0,829\pm 0,095 \text{г}$

\par
С помощью установки на рисунке 1 нашли $r_{max} = 20,2 \pm 0,1$ мм. \par
Рассчитаем величину магнитного момента магнитика по формуле 1:
\[P_{m} = \sqrt{\frac{0,829*981*16,649}{6}} = 47,51\pm 0,1 \text{Эрг/Гс}\] 
Рассчитаем величину намагниченности $p_{m}= \frac{P_{m}}{V} = 496,0 \pm 0,1$Гс\par
Рассчитаем величину магнитного поля по формуле (2): $B_{p} = 4,15 \pm 0,21$кГс (на устройстве примерно 2,5 кГс) \par
Рассчитаем величину остаточной магнитной индукции: $B_{r} = 0,77 \pm 0,29$ Тл (табличное 1,22 Тл)

\subsection{Задание 2}

Измерим зависимость периода свободных колебаний от количества шаров:

\begin{tabular}{|c|c|c|c|c|c|c|}
\hline
Количество шаров & $t_{1}$,с & $t_{2}$,с & $t_{3}$,с&$t_{\text{ср}}$,с & Число колебаний N & Период T, с \\
\hline
12 & 37,50 & 37,48 & 37,49 & 37,49 & 10& 3,50 \\
\hline
10 & 31,20 & 30,73 & 31,25 & 31,06 & 10& 3,10 \\
\hline
8 & 24,68 & 24,73 & 24,55 & 24,65 & 10& 2,46 \\
\hline
6 & 18,50 & 18,65 & 18,64 & 18,60 & 10& 1,86 \\
\hline
4 & 12,90 & 12,60 & 12,80 & 12,76 & 10& 1,28 \\
\hline
\end{tabular}

Построим график:

\begin{figure}[h]
    \centering
    \includegraphics[width=0.8\textwidth]{g1.png}
    \caption{T(n)}
    \label{fig:example}
\end{figure}

Коэффициент наклона МНК равен: k = 0,284\par
Рассчитаем горизонтальную составляющую магнитного поля Земли по формуле:

\[
B_h = \frac{\pi^2 m d^2}{3k^2 P_m} = 0,229 \pm 0,116 \text{Гс}
\]
\newpage
\subsection{Задание 3}
Рассчитаем вертикальную составляющую поля. Для этого определим моменты сил:
\[M = [\vec{P_m}\vec{B}]\]
\[\overrightarrow{B} = \frac{3(\overrightarrow{P_{m}}\overrightarrow{ r})\overrightarrow{r}}{r^{5}}- \frac{\overrightarrow{P_{m}} }{r^{3}}\] 

\begin{tabular}{|c|c|c|c|}
\hline
Количество шаров & Расстояние до маленького кольца & Расстояние до большого кольца & M, дин*см  \\
\hline
4 & - & 0,57 см & 189,92 $\pm 0,03$  \\
\hline
6 & 1,13 см & 0,57 см & 243,06 $\pm 0,04$  \\
\hline
8 & - & 0,57 см & 307,02 $\pm 0,03$  \\
\hline
10 & 0,57 & 0,57 & 365,34 $\pm 0,01$  \\
\hline
12 & - & 1,13 см & 376,03 $\pm 0,03$  \\
\hline

\end{tabular}

Построим график M(n):
\begin{figure}[h]
    \centering
    \includegraphics[width=0.8\textwidth]{g2.png}
    \caption{M(n)}
    \label{fig:example}
\end{figure}

k= 24,725\par
\[B_{v} = 0,520 \pm 0,051 \text{Гс}\]
Магнитное наклонение
\[\beta = arctang\frac{B_{v}}{B_{h}} = 66,2 \pm 3\] 
При $\varphi = 56$ $\beta $ = 71\par
Полная величина индукции магнитного поля:
\[B = \sqrt{B_{v}^2+ B_{h}^2} =  0,568 \text{Гс}\]
По данным из интернета величина магнитной индукции в московской области составляет 0,56 Гс.

\section{Вывод}
Исследовал свойства постоянных магнитов и измерил с их помощью горизонтальную и вертикальную сосотавляющую 
индукции магнитного поля Земли и магнитное наклонение. Итоговое экспериментальное значение 
индукции магнитного поля Земли почти идеально сошлось со значением из интернета. Значение остаточной магнитной индукции 
отличается от справочного около двух раз. Возможно, такое отклонение связано с размагниченностью магнитов, а также их ,,нечистоте,, 
связанное с взятием в жирные руки.

\end{document}