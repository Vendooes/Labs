\documentclass[a4paper]{article}
\usepackage[utf8]{inputenc}
\usepackage[russian,english]{babel}
\usepackage[T2A]{fontenc}
\usepackage[left=10mm, top=20mm, right=18mm, bottom=15mm, footskip=10mm]{geometry}
\usepackage{indentfirst}
\usepackage{amsmath,amssymb}
\usepackage[italicdiff]{physics}
\usepackage{graphicx}
\graphicspath{{images/}}
\DeclareGraphicsExtensions{.pdf,.png,.jpg}
\usepackage{wrapfig}
\usepackage{subcaption}
\usepackage{caption}
\captionsetup[figure]{name=Рисунок}
\captionsetup[table]{name=Таблица}
  
\title{\underline{Отчет о выполненой лабораторной работе 3.2.4 - 3.2.5}}
\author{Воронин Денис, Б04-407}

\begin{document}

\maketitle{}

% \begin{figure}[h!]
% 	\centering
% 	\includegraphics[width=12cm]{mach.PNG}
% 	\caption{Схема установки для исследования вынужденных колебаний}
% 	\label{fig:Holl2}
% \end{figure}

\begin{center}
\textbf{\Large Колебания в электрическом контуре}
\end{center}


\section*{Цель работы:}
Исследование свободных и вынужденных колебаний в колебательном контуре.

\subsection*{В работе используются:}
Осциллограф AKTAKOM ADS-6142H, генератор сигналов специальной формы АКИП-3409/4, магазин сопротивления MCP-60, магазин емкости P5025, магазин индуктивности P567 
типа МИСП, соединительная коробка с шунтирующей емкостью, соединительные одножильные и коаксиальные провода.

\section{Теоретические сведения}

\begin{wrapfigure}{l}{0.4\textwidth}
    \centering
    \includegraphics[width=0.4\textwidth]{p1.png}
    \caption{Последовательный колебательный контур}
    \label{fig:example}
\end{wrapfigure}

\noindent Для RLC контура применим правило Кирхгофа:
\begin{equation}
    RI + U_C + L\frac{dI}{dt} = 0.
\end{equation}
Подставив в уравнение выражение для тока через 1-ое правило Кирхгофа, и разделив обе части уравнения на $CL$, получим:
\begin{equation}
    \frac{d^2U_C}{dt^2} + \frac{R}{L} \frac{dU_C}{dt} + \frac{U_C}{CL}=0
\end{equation}
Произведём замены $\gamma = \frac{R}{2L}$ -- коэффициент затухания, $\omega_0^2 = \frac{1}{LC}$ -- собственная круговая частота, $T_0 = \frac{2\pi}{\omega_0} = 2\pi \sqrt{LC}$ -- период собственных колебаний. Тогда уравнение для напряжения на конденсаторе:
\begin{equation}
    \ddot{U_C} + 2 \gamma \dot{U_C} + \omega_0^2U_C = 0,
\end{equation}
где точкой обозначено дифференцирование по времени. Будем искать решение данного дифференциального уравнения в классе функций следующего вида:
$$U_C(t) = U(t)e^{- \gamma t}.$$
Получим:
\begin{equation}
    \ddot{U} + \omega_1^2 U = 0,
\end{equation}
где
\begin{equation}
    \omega_1^2 = \omega_0^2-\gamma^2
\end{equation}

\subsection{Затухающие колебания}

Для случая $\gamma < \omega_0$ в силу того, что $\omega_1 > 0$, получим:
\begin{equation}
    U_C(t) = U_0 \cdot e^{-\gamma t} \text{cos}(\omega_1 t + \varphi_0).
\end{equation}
Для получения фазовой траектории представим формулу в другом виде:
\begin{equation}
    U_C(t) = e^{-\gamma t}(a \text{cos} \omega_1 t + b \text{sin} \omega_1 t),
\end{equation}
где $a$ и $b$ получаются по формулам:
$$a = U_0 \text{cos} \varphi_0, \qquad b = - U_0 \text{sin} \varphi_0.$$
В более удобном виде запишем выражения для напряжения на конденсаторе и токе через катушку:
\begin{equation}
    U_C (t) = U_{C0} \cdot e^{-\gamma t} (\text{cos} \omega_1 t + \frac{\gamma}{\omega_1} \text{sin} \omega_1 t),
\end{equation}
\begin{equation}
    I(t) = C\dot{U_C}= - \frac{U_{C0}}{\rho} \frac{\omega_0}{\omega_1} e^{-\gamma t} \text{sin} \omega_1 t.
\end{equation}
Введём некоторые характеристики колебательного движения:
\begin{equation}
    \tau = \frac{1}{\gamma} = \frac{2L}{R},
\end{equation}
где $\tau$ -- время затухания (время, за которое амплитуда колебаний уменьшается в $e$ раз).
\begin{equation}
    \Theta = \text{ln} \frac{U_k}{U_{k+1}} = \gamma T_1 = \frac{1}{N_\tau} = \frac{1}{n} \text{ln} \frac{U_k}{U_{k+n}}, 
\end{equation}
где $\Theta$ -- логарифмический декремент затухания, $U_k$ и $U_{k+1}$ -- два последовательных максимальных отклонения величины в одну сторону, $N_\tau$ -- число полных колебаний за время затухания $\tau$.

\subsection{Добротность}

С логарифмическим декрементом связана ещё одна важнейшая характеристика колебательного контура --- его добротность \( Q \):

\begin{equation}
Q = \frac{\pi}{\Theta} = \frac{\pi}{\sqrt{T_1}} = \frac{1}{2} \sqrt{\frac{\omega_0^2}{\gamma^2} - 1} = 
= \frac{1}{2} \sqrt{\frac{R_{\text{кр}}^2}{R^2}} - 1 = \sqrt{\frac{\rho ^2}{R^2} - \frac{1}{4}}.
\end{equation}
где $\rho = \frac{1}{2}R_{cr}=\sqrt{\frac{L}{C}}$\par

Как правило, о добротности говорят, только когда она достаточно велика, то есть \( Q \gg 1 \). Такой добротностью обладают колебательные контуры со слабым затуханием, представляющие большой практический интерес. Для них имеет место сильное неравенство

\begin{equation}
0 < \gamma \ll \omega_0,
\label{eq:2.30}
\end{equation}

или, в терминах параметров контура,

\begin{equation}
0 < R \ll R_{\text{кр}} = 2\rho.
\label{eq:2.31}
\end{equation}

Малость отношения $\frac{\gamma }{\omega_{0}}$ дает следующее:
\begin{equation}
U_C(t) \approx U_0 e^{-\gamma t} [\cos \omega_0 t + \frac{\gamma}{\omega_0} \sin \omega_0 t],
\end{equation}

\begin{equation}
I(t) \approx -\frac{U_0}{\rho} e^{-\gamma t} \sin \omega_0 t, 
\end{equation}

где \( U_0 \approx U_{C_0} \), а добротность \( Q \) связать с характеристиками контура соотношениями

\begin{equation}
Q \approx \frac{\omega_0}{2\gamma} = \frac{\pi}{\gamma T_0} = \frac{\tau \omega_0}{2} = \frac{1}{R} \sqrt{\frac{L}{C}} = \frac{\rho}{R} \gg 1.
\end{equation}

\subsection{Вынужденные колебания}

\begin{wrapfigure}{l}{0.4\textwidth}
    \centering
    \includegraphics[width=0.25\textwidth]{p2.png}
    \caption{Последовательный контур с внешней ЭДС}
    \label{fig:example}
\end{wrapfigure}

Теперь рассмотрим случай \textit{вынужденных колебаний} под действием внешней внешнего синусоидального источника. 


Для этого воспользуемся методом \textit{комплексных амплитуд} для схемы на рисунке:
\begin{equation}
    \ddot{I} + 2 \gamma \dot{I} + \omega^2 I = - \varepsilon \frac{\Omega}{L} e^{i\Omega t}.
\end{equation}
Решая данное дифференциальное уравнение получим решение:
\begin{equation}
    I = B\cdot e^{-\gamma t} \text{sin}(\omega t - \Theta) + \frac{\varepsilon_0 \Omega}{L \phi_0} \text{sin} (\Omega t - \varphi).
\end{equation}
Нетрудно видеть, что частота резонанса будет определяться формулой:
\begin{equation}
    \omega_0 = \frac{1}{2 \pi \sqrt{LC}}.
\end{equation}
\subsection{Способы изменения добротности}
Способы измерения добротности $Q = \dfrac{W_0}{W_{loss,\,\tau}} = \dfrac{\pi}{\Theta}$:
\begin{enumerate}
    \item с помощью потери амплитуды свободных колебаний: 
    \begin{equation}
        \Theta = \frac{1}{n} \text{ln}\frac{U_k}{U_{k+n}},
    \end{equation}
    \item с помощью амплитуды резонанса можно получить добротность (в координатах $U_C/U_0$, где $U_0$ -- амплитуда колебаний напряжения источника, от частоты генератора). Отсюда нетрудно определить декремент затухания $\gamma = \frac{\omega_0}{2Q}$,
    \item с помощью среза АЧХ на уровне 0.7 от максимальной амплитуды, тогда <<дисперсия>> ($\Delta \Omega$) будет численно равна коэффициенту $\gamma$, то есть $Q = \frac{\nu_0}{2 \Delta \Omega}$.
    \item с помощью нарастания амплитуд в вынужденных колебаниях:
    \begin{equation}
        \Theta = \frac{\omega_0 n}{2\text{ln} \frac{U_0 - U_k}{U_0 - U_{k+n}}}.
    \end{equation}
    \item  с помощью формулы\begin{equation}
        \Theta = \frac{1}{R}\sqrt{\frac{L}{C}}
    \end{equation}
\end{enumerate}

\section{Экспериментальная установка}

Колебательный контур состоит из постоянной индуктивности $L$, активного сопротивления $R$, переменной ёмкости $C$ и сопротивления $R$. Картина колебаний напряжения на ёмкости наблюдается на экране двухканального осциллографа.

\begin{figure}[h]
    \centering
    \includegraphics[width=0.8\textwidth]{mach.png}
    \caption{Схема установки}
    \label{fig:my_image}
\end{figure}

Для возбуждения затухающих колебаний используется генератор сигналов специальной формы. Сигнал с генератора поступает через конденсатор $C_1$ на вход колебательного контура. Данная ёмкость необходима чтобы выходной импеданс генератора был много меньше импеданса колебательного контура и не влиял на процессы, проходящие в контуре.

Установка предназначена для исследования не только возбужденных, но и свободных колебаний в электрической цепи. При изучении свободно затухающих колебаний генератор специальных сигналов на вход колебательного контура подает периодические короткие импульсы, которые заряжают конденсатор $C$.
За время между последовательными импульсами происходит разрядка конденсатора через резистор и катушку индуктивности.
Напряжение на конденсаторе $U_C$ поступает на вход канала 1(X) электронного осциллографа. Для наблюдения фазовой картины затухающих колебаний на канал 2(Y) подается напряжение с резистора $R$ (пунктирная линия на схеме установки), которое пропорционально току $I$ ($I \propto dU_C/dt$).

При изучении возбужденных колебаний на вход колебательного контура подается синусоидальный сигнал. С помощью осциллографа возможно измерить зависимость амплитуды возбужденных колебаний в зависимости от частоты внешнего сигнала, из которого возможно определить 
добротность колебательного контура. Альтернативным способом расчета добротности контура является определение декремента затухания по картине установления возбужденных колебаний. В этом случае генератор сигналов используется для подачи пугов синусоидальной формы.


\newpage

\section{Ход работы}

\subsection{Измерение периодов свободных колебаний}

Определим с помощью осциллографа период затухающих колебаний:
\[T= 74 \text{мкс}\]

По периоду колебаний определим нулевую емкость колебательного контура:
\[C_{0} = \frac{T^2}{4\pi^{2}L_{0}} = 1,39 \pm 0,15 \text{нФ}\]

Проведем измерение периодов:

\begin{tabular}{|c|c|c|c|c|c|}
\hline
Емкость, C[мкФ] & 0 & 0,002 & 0,004 & 0,006 & 0,008 \\
\hline
Период T,мc  & 0,074 & 0,111 & ,141 & 0,167 & 0,191 \\
\hline
\end{tabular}

\subsection{Критическое сопротивление и декремент затухания}

Рассчитаем емкость $C^{*}$:
\[C_{*} = \frac{1}{4\pi^{2}L\nu_{0}^{2}} = (6,01\pm 0,03)*10^{-9} \text{Ф}\]
Рассчитаем $R_{\text{кр}}$ по формуле:
\[R_{\text{кр}} = 2\sqrt{\frac{L}{C^{*}}} = 8165,1 \pm 0,4 \text{Ом}\]

Определим сопротивление, при котором колебательный режим переходит в апереодический: $R_\text{апереодический} = 3500$ Ом.

Измерим декремент затуханий $\Theta $ для разный сопротивлений:

\begin{tabular}{|c|c|c|c|c|}
\hline
R, Ом & $U_{m}$, В &$U_{m+n}$, В & n & $\Theta $ \\
\hline
408 & 86 & 56,4 & 2 & $0,21 \pm 0,03$\\
\hline
735 & 99 & 38 & 3 & $0,32 \pm 0,03$ \\
\hline
1061 & 92,4 & 26 & 3 & $0,42 \pm 0,03$ \\
\hline
1388 & 86,8 & 19,2 & 3 & $0,50 \pm 0,03$ \\
\hline
1715 & 81,2& 13,2 & 3 & $0,61 \pm 0,03$ \\
\hline
2041 & 74 & 9,2 & 3 & $0,69 \pm 0,03$ \\
\hline
\end{tabular}

Зафиксируем два различных значения:$R_{1} = 408$ Ом, $R_{2} = 2041$ Ом.

\subsection{Свободные колебания на фазовой плоскости}
Добъемся на экране осциллографа каритинки спирали\par
Измерим декремент затуханий $\Theta $ для этого случая:

\begin{tabular}{|c|c|c|c|c|}
\hline
R, Ом & $O_{x}$, кл &$O_{x+n}$, кл & n & $\Theta $ \\
\hline
408 & 4 & 3,6 & 1 & $0,11 \pm 0,04$\\
\hline
735 & 5 & 4 & 1 & $0,22 \pm 0,04$ \\
\hline
1061 & 4 & 2,8 & 1 & $0,36 \pm 0,04$ \\
\hline
1388 & 3,3 & 2 & 1 & $0,50 \pm 0,04$ \\
\hline
1715 & 2,6& 1,5 & 1 & $0,55 \pm 0,04$ \\
\hline
2041 & 4,5 & 2 & 2 & $0,71 \pm 0,01$ \\
\hline
\end{tabular}

\subsection{Исследование резонансных кривых}

Изменяя частоту генератора вблизи резонансной частоты и наблюдая синусоиду
на первом канале на экране осциллографа, убедимся, что амплитуда
колебаний максимальна при достижении резонансной частоты. Ее
значение $U_{amp} = 2,56$ В.\par
Снимем АЧХ и ФЧХ колебательного контура вблизи резонанса:\par

В сторону возрастания $\longrightarrow $:




\begin{tabular}{|c|c|c|c|c|c|c|c|c|c|c|c|}
    \hline
    $\nu $, Гц & 4226 & 4423 & 4616 & 4811 & 5006 & 5201 & 5336 & 5591 & 5786 & 5998& 6176 \\
    \hline
    АЧХ, В & 1,02 & 1,16 & 1,32 & 1,49 & 1,67 & 1,87 & 2,08 & 2,26 & 2,48 & 2,54 & 2,56 \\
    \hline
    ФЧХ, мкс & 74 & 72 & 66 & 62 & 58 & 50 & 46 & 39 & 35 & 31 & 27 \\
    \hline
\end{tabular}

В сторону убывания $\longleftarrow $:

\begin{tabular}{|c|c|c|c|c|c|c|c|c|c|c|c|}
    \hline
    $\nu $, Гц & 6176 & 6371 & 6466 & 6761 & 6958 & 7151 & 7346 & 7541 & 7736 & 7931& 8126 \\
    \hline
    АЧХ, В & 2,56 & 2,54 & 2,52 & 2,46 & 2,34 &2,26 & 2,20 & 2,12 & 2,04 & 1,96 & 1,92 \\
    \hline
    ФЧХ, мкс & 27 & 24 & 18 & 16 & 15 & 11 & 10 & 8 & 7 & 6 & 6 \\
    \hline
\end{tabular}
\newpage

\subsection{Процессы установления и затухания}
Измерим амлитуды колебаний для $R_{1}$:

\begin{tabular}{|c|c|}
    \hline
    \textbf{Возрастание $\nearrow $} & \textbf{Убывание $\searrow $} \\
    \hline
    $U_{k} = 4,5$ В & $U_{k} = 6,9$ В \\
    \hline
    $U_{k+3} = 6,9$ В & $U_{k+3} = 2,1$ В \\
    \hline
    $U_{k+1} = 2,7$ В & $U_{k+1} = 2,2$ В \\
    \hline
    $U_{k+6} = 7,1$ В & $U_{k+6} = 4,10$ В \\
    \hline
\end{tabular}
\par
Измерим амлитуды колебаний для $R_{1}$:

\begin{tabular}{|c|c|}
    \hline
    \textbf{Возрастание $\nearrow $} & \textbf{Убывание $\searrow $} \\
    \hline
    $U_{k} = 1,76$ В & $U_{k} = 2,06$ В \\
    \hline
    $U_{k+3} = 2,06$ В & $U_{k+1} = 1,72$ В \\
    \hline
    $U_{k+2} = 2,0$ В & $U_{k+2} = 3,40$ В \\
    \hline

\end{tabular}

Рассчитаем логарифмический декремент затухания и добротность для $R_{1}$ и $R_{2}$ по формуле 
\begin{equation}
\Theta = \frac{1}{n} \ln \frac{U_0 - U_k}{U_0 - U_{k+n}}.
\end{equation}

Для $R_{1}$ имеем:
\begin{tabular}{|c|c|c|c|}
    \hline
     &\textbf{$\Theta \nearrow$ }&\textbf{$\Theta \searrow$ } & \textbf{Q} \\
    \hline
    1 пара &0,27 &0,31 & $8,07\pm 0,83$ \\
    \hline
    2 пара & 0,32&0,33 & $8,09\pm 0,83$ \\
    \hline
    3 пара & 0,31&0,34& $8,08\pm 0,83$ \\
    \hline

\end{tabular}

Для $R_{2}$ имеем:

\begin{tabular}{|c|c|c|c|}
    \hline
     &\textbf{$\Theta \nearrow$ }&\textbf{$\Theta \searrow$ } & \textbf{Q} \\
    \hline
    1 пара &1,23 &1,20 & $2,23\pm 0,24$ \\
    \hline
    2 пара & 1,19&1,21 & $2,21\pm 0,24$ \\
    \hline
    3 пара & 1,18&1,19& $2,20\pm 0,24$\\
    \hline

\end{tabular}

\section{Обработка результатов}

Построим график $T_{exp} = f(T_{theor})$:

\begin{figure}[h]
    \centering
    \includegraphics[width=0.8\textwidth]{p3.png}
    \caption{$T_{exp}(T_{theor})$}
    \label{fig:my_image}
\end{figure}
\newpage

Из графика видно, что результаты сходятся, погрешность $< 1\%$.

Построим график $1/Q^2 = f(1/R^2)$.

\begin{figure}[h]
    \centering
    \includegraphics[width=0.8\textwidth]{p5.png}
    \caption{$1/Q^2 = f(1/R^2)$}
    \label{fig:my_image}
\end{figure}

Коэффициент затухания $K = 1492115 \pm 70000$ $ \text{Ом}^2$.\par


Зная коэффициент затухания, найдём $R_{\text{кр}}$ по формуле $R_{\text{кр}} = 2\pi \sqrt{K} = 7671 \pm 200$ Ом, 
это близко к теоретическому значению $R_{\text{кр}} = 8165$ Ом.

 Расчитаем добротность для максимального и минимального значения \(\theta\) и теоретическое с теми же параметрами.


Вычисление добротности контура по секции 3.2:
  \begin{equation}
  Q(\theta_{\text{min}}) = 8.97 \quad Q(\theta_{\text{max}}) = 2.40
  \end{equation}

 Вычисление добротности контура по секции 3.3:
  \begin{equation}
  Q(\theta_{\text{min}}) = 8.49 \quad Q(\theta_{\text{max}}) = 2.21
  \end{equation}

 Вычисление добротности контура теоретически:
  \begin{equation}
  Q(\theta_{\text{min}}) = 9.16 \quad Q(\theta_{\text{max}}) = 2.34
  \end{equation}


По секции 3.4 построим АЧХ в масштабе \(U/U_{\text{res}} = f(\nu/\nu_{\text{res}})\)

\begin{figure}[h]
    \centering
    \includegraphics[width=0.5\textwidth]{p6.png}
    \caption{$U/U_{\text{res}}(\nu/\nu_{\text{res}})$}
    \label{fig:my_image}
\end{figure}

Рассматриваем добротность по формуле $Q = \nu_{\text{рез}}/2\Delta\nu$, $Q = 8,08$.
\newpage

Определение добротности по графика АЧХ

\begin{figure}[h]
    \centering
    \includegraphics[width=0.5\textwidth]{p7.png}
    \caption{}
    \label{fig:my_image}
\end{figure}


\begin{tabular}{|c|c|c|}
\hline
$R$, Ом & $\dfrac{2\Delta\omega}{\omega_0}$ & $Q$ \\
\hline
408 & 0,12 & $8,21\pm 0,65$ \\
\hline
1633 & 0,43 & $2,33 \pm 0,13$ \\
\hline
\end{tabular}


Опеределение добротности по ФЧХ:

\begin{figure}[h]
    \centering
    \includegraphics[width=0.5\textwidth]{p8.png}
    \caption{}
    \label{fig:my_image}
\end{figure}


\begin{tabular}{|c|c|c|}
\hline
$R$, Ом & $\dfrac{\Delta\omega}{\omega_0}$ & $Q$ \\
\hline
408 & 0,115 & $8,15 \pm 0,68$ \\
\hline
1633 & 0,415 & $2,41 \pm 0,15$ \\
\hline
\end{tabular}

Отог:



\begin{table}[h]
\centering
\begin{tabular}{|c|c|c|c|c|c|}
\hline
$R$, Ом & $f(L, C, R)$ & $f(\theta)$ & Фаз. сдвиг & АЧХ & ФЧХ \\
\hline
408 & $9,13 \pm 0,10$  & $8,23 \pm 0,65$  & $8,19 \pm 0,86$  & $8,21 \pm 0,65$  & $8,15 \pm 0,68$  \\
\hline
1633 & $2,33 \pm 0,01$  & $2,42 \pm 0,06$  & $2,31 \pm 0,18$  & $2,33 \pm 0,13$  & $2,41 \pm 0,15$  \\
\hline
\end{tabular}
\end{table}

\section{Вывод}

В данной лабораторной работе мы исследовали свободные и вынужденные колебания в электрическом контуре и различными способами находили его добротность. Достаточно эффективен способ вычисления через декремент затухания. Фазовая спираль даёт высокую погрешность,
 поэтому это не очень надежный способ вычисления добротности. Способы вычисления через АЧХ и ФЧХ пригодны для использования, если есть специальная программа, позволяющая вычислять ширину резонансной кривой, и хорошо снятые данные, которые не всегда удается точно измерить. Несмотря на то, что при нашей оценке у $R_2 = 1668 \, \text{Ом}$ относительная погрешность в этих опытах примерно около 10 процентов, 
, учитывая наши нкачествнные данные(часть которых принадлежит другой установке) мы получили хороший результат.


\end{document}