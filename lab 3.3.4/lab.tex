\documentclass[a4paper]{article}
\usepackage[utf8]{inputenc}
\usepackage[russian,english]{babel}
\usepackage[T2A]{fontenc}
\usepackage[left=10mm, top=20mm, right=18mm, bottom=15mm, footskip=10mm]{geometry}
\usepackage{indentfirst}
\usepackage{amsmath,amssymb}
\usepackage[italicdiff]{physics}
\usepackage{graphicx}
\graphicspath{{images/}}
\DeclareGraphicsExtensions{.pdf,.png,.jpg}
\usepackage{wrapfig}
\usepackage{subcaption}
\usepackage{caption}
\captionsetup[figure]{name=Рисунок}
\captionsetup[table]{name=Таблица}
  
\title{\underline{Отчет о выполненой лабораторной работе 3.3.4}}
\author{Воронин Денис, Б04-407}

\begin{document}

\maketitle

\begin{center}
\textbf{\Large Эффект Холла в полупроводниках}
\end{center}

\textbf{Цель работы:} измерение подвижности и концентрации носителей заряда
 в полупроводниках.

\textbf{В работе используются:}  электромагнит с источником питания GPR; цифровой вольтметр В7-65/5; батарейка 1,5~В; реостат; миллиамперметр; 
образцы легированного германия; измеритель магнитной индукции АТЕ-8702.

\section{\text{Теоретические сведения}}

Во внешнем магнитном поле $\vec{B}$ на заряды действует сила Лоренца:
\begin{equation}
\vec{F} = q\vec{E} + q\vec{u} \times \vec{B}.
\label{eq:lorentz}
\end{equation}

Эта сила вызывает движение носителей, направление которого в общем случае не совпадает с $\vec{E}$. Действительно, траектории частиц будут либо искривляться, либо, если геометрия проводника этого не позволяет, возникнет дополнительное электрическое поле, компенсирующее магнитную составляющую силы Лоренца.
Возникновение поперечного тока электрического поля в образце, помещённом во внешнее магнитное поле, называют \textbf{эффектом Холла}.

Связь между электрическим полем $\vec{E}$ и плотностью тока $\vec{j}$ в условиях эффекта Холла уже не может быть описана скалярным коэффициентом проводимости $\sigma$. Тем не менее закон Ома можно по-прежнему записать в форме
\begin{equation}
\vec{j} = \hat{\sigma} \vec{E},
\label{eq:ohm_tensor}
\end{equation}
если под $\hat{\sigma}$ понимать \textbf{тензор проводимости}. В заданном базисе он представляется матрицей $3 \times 3$:
\begin{equation}
\vec{j} = \hat{\sigma} \vec{E} \equiv 
\begin{pmatrix}
\sigma_{xx} & \sigma_{xy} & \sigma_{xz} \\
\sigma_{yx} & \sigma_{yy} & \sigma_{yz} \\
\sigma_{zx} & \sigma_{zy} & \sigma_{zz}
\end{pmatrix}
\vec{E}.
\label{eq:sigma_matrix}
\end{equation}
или
\begin{equation}
j_i = \sum_k \sigma_{ik} E_k, \quad \text{где } i,k = \{x,y,z\}.
\label{eq:ohm_index}
\end{equation}

\begin{wrapfigure}{l}{0.5\textwidth}
  \centering
  \includegraphics[width=0.48\textwidth]{p1.png}
  \caption{Силы, действующие на положительный носитель заряда в проводя
 щей среде при наличии магнитного поля}
  \label{fig:example}
\end{wrapfigure}

По оси $x$ носители будут двигаться так, как если бы магнитного поля не было: $j_x = \sigma_0 E_x$ ($j_y = j_z = 0$), где $\sigma_0 = q n \mu$ — удельная проводимость среды в отсутствие $\vec{B}$.

Выразим общую связь между $\vec{E}$ и $\vec{j}$ для случая носителей одного типа. Магнитное поле по-прежнему направим вдоль оси $z$, а о направлении $\vec{E}$ и $\vec{j}$ никаких предположений делать не будем. При движении носителей с постоянной средней скоростью сила Лоренца будет уравновешена трением со стороны среды:
\begin{equation}
q(\vec{E} + \vec{u} \times \vec{B}) - \frac{q\vec{u}}{\mu} = 0.
\label{eq:force_balance}
\end{equation}

С учётом введённых выше обозначений этот баланс сил можно переписать как
\begin{equation}
\vec{E} = \frac{\vec{j}}{\sigma_0} - \frac{1}{nq} \vec{j} \times \vec{B}.
\label{eq:general_ohm_hall}
\end{equation}

Полученное соотношение можно назвать \textbf{обобщённым законом Ома} при наличии внешнего магнитного поля. Второе слагаемое в правой части как раз отвечает эффекту Холла — возникновению поперечного направлению тока электрического поля.

\newpage

\subsection{Мостик Холла}

\begin{wrapfigure}{l}{0.5\textwidth}
  \centering
  \includegraphics[width=0.35\textwidth]{p2.png}
  \caption{Схема для исследования влияния магнитного поля на проводящие свойства: мостик Холла}
  \label{fig:example}
\end{wrapfigure}

Сила Лоренца, действующая со стороны перпендикулярного пластинке магнитного поля, <<прибивает>> носители заряда к краям образца, что создаёт холловское электрическое поле, компенсирующее эту силу. Поперечное напряжение между краями пластинки (\textbf{холловское напряжение}) равно $U_\perp = E_y a$, где, согласно уравнению:
\begin{equation}
E_y = \rho_{yx} \cdot j_x = \frac{j_x B}{nq}.
\label{eq:hall_field}
\end{equation}

Плотность тока, текущего через образец, равна $j_x = I / ah$, где $I$ — полный ток, $ah$ — поперечное сечение. Таким образом, для холловского напряжения имеем
\begin{equation}
U_\perp = \frac{B}{nqh} \cdot I = R_\mathrm{H} \cdot \frac{B}{h} \cdot I,
\label{eq:hall_voltage}
\end{equation}
где константу
\begin{equation}
R_\mathrm{H} = \frac{1}{nq}
\label{eq:hall_constant}
\end{equation}

называют \textbf{постоянной Холла}. Знак постоянной Холла определяется знаком заряда носителей.

Продольная напряжённость электрического поля равна
\begin{equation}
E_x = \rho_{xx} \cdot j_x = j_x / \sigma_0,
\label{eq:longitudinal_field}
\end{equation}
и падение напряжения $U_\parallel = E_x l$ вдоль пластинки определяется омическим сопротивлением образца $R_0 = l / (\sigma_0 a h)$:
\begin{equation}
U_\parallel = I R_0.
\label{eq:longitudinal_voltage}
\end{equation}

Интересно отметить, что несмотря на то, что тензор проводимости явно зависит от $B$, продольное сопротивление образца в данной геометрии от магнитного поля \textbf{не зависит}.



\begin{figure}[h]
    \centering
    \includegraphics[width=0.6\textwidth]{p3.png}
    \caption{Схема установки}
    \label{fig:example}
\end{figure}

В зазоре электромагнита (рис.~1a) создаётся постоянное магнитное поле, величину которого можно менять с помощью регуляторов источника питания электромагнита. Ток питания электромагнита измеряется амперметром источника питания A$_1$. Разъём K$_1$ позволяет менять направление тока в обмотках электромагнита.

В образце с током, помещённом в зазор электромагнита, между контактами 3 и 4 возникает разность потенциалов $U_{34}$,
которая измеряется с помощью цифрового вольтметра.

Можно исключить влияние омического падения напряжения иначе, если при каждом токе через образец измерять напряжение между точками 3 и 4 в отсутствие магнитного поля. 
При фиксированном токе через образец это дополнительное к ЭДС Холла напряжение $U_0$ остаётся неизменным. От него следует (с учётом знака) отсчитывать величину ЭДС Холла: $\mathcal{E}_x = U_{34} \pm U_0$. При таком способе измерения нет необходимости проводить повторные измерения с противоположным направлением магнитного поля.

\newpage

\section{{Ход работы}}

\subsection{Градуировка электромагнита}


\begin{table}[h]
\begin{tabular}{|c|c|c|c|c|c|c|c|}
\hline
Сила тока I, А & 0 & 0,1 & 0,2 & 0,3 & 0,4 & 0,5 & 0,6 \\ 
\hline
Магнитная индукция B, мТл  & 21,1  & 110,7  & 189,6  & 294,4  &402,1  & 503,9  & 598,3  \\
\hline
\end{tabular}
\caption{Градуировка}
\label{tab:my_table}
\end{table}


\subsection{Измерение ЭДС Холла}

Проведем измерения $U_{34} = f(I_\mathrm{M})$ при постоянном токе через образец 
(всего 6--8 серий для токов в интервале 0{,}2--1~мА). При каждом новом значении тока через образец величина $U_0$ будет иметь своё значение.

\begin{table}[h]
\begin{tabular}{|c|c|c|c|c|c|c|c|}
\hline
Сила тока $I_{M}$, А & 0 & 0,15 & 0,3 & 0,45 & 0,6 & 0,75 & 0,9 \\ 
\hline
Напряжение, $U_{34}$, мВ  & -1,98  & -1,33  & -0,71  & -0,27  &-0,091  & -0,71  & -0,69  \\
\hline
\end{tabular}
\caption{I = 200 мкА, U = -2,09 мВ}
\label{tab:my_table}

\begin{tabular}{|c|c|c|c|c|c|c|c|}
\hline
Сила тока $I_{M}$, А & 0 & 0,15 & 0,3 & 0,45 & 0,6 & 0,75 & 0,9 \\ 
\hline
Напряжение, $U_{34}$, мВ&-2,09  & -1,99 & -1,12   &-0,42   &0,13     &0,67   &1,02   \\
\hline
\end{tabular}
\caption{I = 300 мкА, U = -3,15 мВ}
\label{tab:my_table}

\begin{tabular}{|c|c|c|c|c|c|c|c|}
\hline
Сила тока $I_{M}$, А & 0 & 0,15 & 0,3 & 0,45 & 0,6 & 0,75 & 0,9 \\ 
\hline
Напряжение, $U_{34}$, мВ & -4 & -2,67  &   -1,46&-0,56   &0,18     &0,9   &1,35   \\
\hline
\end{tabular}
\caption{I = 400 мкА, U = -4,12 мВ}
\label{tab:my_table}

\begin{tabular}{|c|c|c|c|c|c|c|c|}
\hline
Сила тока $I_{M}$, А & 0 & 0,15 & 0,3 & 0,45 & 0,6 & 0,75 & 0,9 \\ 
\hline
Напряжение, $U_{34}$, мВ& -5,01  &-3,35   &-1,85   &-0,69   &0,23     &1,05   &1,77   \\
\hline
\end{tabular}
\caption{I = 500 мкА, U = -5,29 мВ}
\label{tab:my_table}

\begin{tabular}{|c|c|c|c|c|c|c|c|}
\hline
Сила тока $I_{M}$, А & 0 & 0,15 & 0,3 & 0,45 & 0,6 & 0,75 & 0,9 \\ 
\hline
Напряжение, $U_{34}$, мВ&-6,02  &-3,96   &-2,19   &-0,82   &  0,27   & 1,25  &2,07   \\
\hline
\end{tabular}
\caption{I = 600 мкА, U = -6,95 мВ}
\label{tab:my_table}

\begin{tabular}{|c|c|c|c|c|c|c|c|}
\hline
Сила тока $I_{M}$, А & 0 & 0,15 & 0,3 & 0,45 & 0,6 & 0,75 & 0,9 \\ 
\hline
Напряжение, $U_{34}$, мВ& -7,09 & -4,67  &-2,59   &-1,02   &0,29     &1,43   &2,51   \\
\hline
\end{tabular}
\caption{I = 700 мкА, U = -7,4 мВ}
\label{tab:my_table}

\begin{tabular}{|c|c|c|c|c|c|c|c|}
\hline
Сила тока $I_{M}$, А & 0 & 0,15 & 0,3 & 0,45 & 0,6 & 0,75 & 0,9 \\ 
\hline
Напряжение, $U_{34}$, мВ&-8,02  &-5,34   &-2,96   &-1.04   &0,35     &1,64   &2,77   \\
\hline
\end{tabular}
\caption{I = 800 мкА, U = -8,44 мВ}
\label{tab:my_table}
\end{table}

При максимальном токе через образец ($\approx 1$~мА) проведем измерения $U_{34} = f(I_\mathrm{M})$ при другом направлении магнитного поля через образец (используем значения из последнего измерения).


\begin{tabular}{|c|c|c|c|c|c|c|c|}
\hline
Сила тока $I_{M}$, А & 0 & 0,15 & 0,3 & 0,45 & 0,6 & 0,75 & 0,9 \\ 
\hline
Напряжение, $U_{34}$, мВ&-8,85  &-11,68   &-14,36   &-16,46   &-18,32     &-19,96 &21,34   \\
\hline
\end{tabular}

\subsection{Определение характера проводимости}

\begin{figure}[h]
    \centering
    \includegraphics[width=0.4\textwidth]{p4.jpg}
    \caption{Схема установки}
    \label{fig:example}
\end{figure}

По правилу векторного произведения определили направление, в котором летят частицы. Из направления выяснили, что
они положительные.

\subsection{Определение удельной проводимости}
При токе 1мА измерили падение напряжения на клеммах 3,4 установки: $U_{3,4} = 0,153$ В.\par

Характеристики образца:$L_{3,5} = 15 mm$, l = 8 mm, a = 2 mm.

\section{Обработка результатов}

Построим график зависимости B от I.

\begin{figure}[h]
    \centering
    \includegraphics[width=0.6\textwidth]{f1.png}
    \caption{Зависимость магнитной индукции B от силы тока I}
    \label{fig:example}
\end{figure}

Рассчитаем ЭДС Холла и построим на одном листе семейство прямых $\varepsilon_{x} = f(B)$

\begin{figure}[h]
    \centering
    \includegraphics[width=0.6\textwidth]{f2.png}
    \caption{Зависимость Холловского напряжения от магнитной индукции}
    \label{fig:example}
\end{figure}


\begin{tabular}{c c c}
\hline
Ток, мА & Угловой коэффициент, мВ/мЛ & $R^2$ \\
\hline
200 & 0.003919 & 0.9853 \\
300 & 0.006181 & 0.9785 \\
400 & 0.007671 & 0.9729 \\
500 & 0.009127 & 0.9761 \\
600 & 0.01078  & 0.9726 \\
700 & 0.012197 & 0.9766 \\
800 & 0.012817 & 0.9766 \\
\hline
\end{tabular}


\begin{figure}[h]
    \centering
    \includegraphics[width=0.6\textwidth]{f3.png}
    \caption{Угловой коэффициент $k = \frac{d\varepsilon_{Холла}}{dB}$, мВ/мТл}
    \label{fig:example}
\end{figure}

Рассчитаем постоянную Холла:
\[R_{\text{х}} = a*k =  0,015*0,002 = (30 \pm 2) * 10^{-6} \frac{\text{м}^3}{\text{Кл}}\] 

Определим концентрацию зарядов:
\[n = \frac{1}{R_{x}*q} = (2,01 \pm 0,3)*10^{23}\]

Определим удельную проводимость.\par
Формула:
\[
\sigma = \frac{I \cdot L_{35}}{U_{35} \cdot a \cdot l}
\]

Подставим значения:
\[
\sigma = \frac{10^{-3} \cdot 0.015}{0.153 \cdot 0.002 \cdot 0.008} =  6.127 \pm 0,048
\]


По формуле
\begin{equation}
b = \frac{\sigma}{en} = \sigma R_x
\label{eq:mobility_formula}
\end{equation}
вычислим подвижность носителей тока в образце:
\begin{equation}
b = 0.021 \pm 0.001~\frac{\mathrm{м}^2}{\mathrm{B} \cdot \mathrm{с}}.
\label{eq:mobility_value}
\end{equation}

\begin{table}[h]
\centering
\begin{tabular}{|c|c|c|c|c|}
\hline
$\bar{R}_X \pm \Delta R_X$ & $v$ & $n \pm \Delta n$ & $\sigma \pm \Delta \sigma$ & $b \pm \Delta b$ \\
$10^{-6}\,\text{м}^3/\text{Кл}$ & Знак носит. & $(\text{м}^3)^{-1}$ & $(\Omega \cdot \text{м})^{-1}$ & $\text{см}^2/(\text{В} \cdot \text{с})$ \\
\hline
$30\pm 2$& положительный &$(2,01 \pm 0,3)*10^{23}$  &$6.127 \pm 0,048$ & $0.021 \pm 0.001$\\
\hline
\end{tabular}
\end{table}
\textbf{\Large Вывод}: измерил подвижности и концентрации носителей заряда в полупроводниках.
Результаты работы хорошо совпали с табличными значениями, особенно концентрация. Постоянная Холла получилась меньше, чем табличные данные.
Это связано с тем, что мы не очень качественно провели градуировку, что видно также на графике.

\end{document}