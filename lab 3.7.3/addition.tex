\documentclass[a4paper]{article}
\usepackage[utf8]{inputenc}
\usepackage[russian,english]{babel}
\usepackage[T2A]{fontenc}
\usepackage[left=10mm, top=20mm, right=18mm, bottom=15mm, footskip=10mm]{geometry}
\usepackage{indentfirst}
\usepackage{amsmath,amssymb}
\usepackage[italicdiff]{physics}
\usepackage{graphicx}
\graphicspath{{images/}}
\DeclareGraphicsExtensions{.pdf,.png,.jpg}
\usepackage{wrapfig}
\usepackage{subcaption}
\usepackage{caption}
\captionsetup[figure]{name=Рисунок}
\captionsetup[table]{name=Таблица}


\begin{document}
Результаты определения резонансных частот для синусоидальных колебаний \par
\begin{tabular}{|c|c|c|c|c|c|c|c|}
\hline
n & 1 & 2 & 3 & 4 & 5 & 6 & 7 \\
\hline
$v_n$, (согласованная линия) МГц & 3.9 & 7.8 & 11.7 & 15.7 & 19.6 & 23.5 & 27.5 \\
\hline
$v_n$, (без нагрузки) МГц & 4 & 8 & 12 & 16 & 20 & 24 & 28 \\
\hline
\end{tabular}

\begin{figure}[h!]
	\centering
	\includegraphics[width=11cm]{a1.PNG}
	\caption{График зависимости резонансных частот от их номера}
	\label{fig:Holl2}
\end{figure}

\begin{figure}[h!]
	\centering
	\includegraphics[width=11cm]{a2.PNG}
	\caption{График зависимости резонансных частот от их номера}
	\label{fig:Holl2}
\end{figure}

 Из графика по данным с согласованной нагрузкой можно получить информацию о угле наклона и фазовой скорости.

\begin{equation*}
\begin{aligned}
k_1 &= \frac{V_\Phi}{l} = (4 \pm 0.003)~\text{МГц} \\
V_\Phi &= (2.012 \pm 0.002) * 10^{10}~\text{см/с}
\end{aligned}
\end{equation*}

А для сигнала без нагрузки:

\begin{equation*}
\begin{aligned}
k_2 &= \frac{V_\Phi}{l} = (3.93 \pm 0.006)~\text{МГц} \\
V_\Phi &= (1.977 \pm 0.003) * 10^{10}~\text{см/с}
\end{aligned}
\end{equation*}
Для групповой скорости
\begin{equation*}
v_n = \frac{V_{\mathrm{rp}}}{l} (n + n_0)
\end{equation*}

Приведем данные и график полученный по этим данным

\begin{tabular}{|c|c|c|c|c|c|}
\hline
n & 1 & 2 & 3 & 4 & 5 \\
\hline
$v_n$, (согласованная линия) МГц & 3.92 & 7.82 & 11.74 & 15.65 & 19.59 \\
\hline
$v_n$, (без нагрузки) МГц & 3.9 & 7.8 & 11.7 & 15.7 & 19.5 \\
\hline
\end{tabular}

\begin{figure}[h!]
	\centering
	\includegraphics[width=11cm]{a3.PNG}
	\caption{График зависимости резонансных частот от их номера для прямоугольных сигналов}
	\label{fig:Holl2}
\end{figure}

\begin{figure}[h!]
	\centering
	\includegraphics[width=11cm]{a4.PNG}
	\caption{График зависимости резонансных частот от их номера для прямоугольных сигналов}
	\label{fig:Holl2}
\end{figure}

Для данных графиков получаем приведенные значения снизу. Для согласованной линии получаем:

\begin{equation*}
\begin{aligned}
k_1 &= \frac{V_\Phi}{l} = (3.917 \pm 0.003)~\text{МГц} \\
V_\Phi &= (1.97 \pm 0.02) * 10^{10}~\text{см/с}
\end{aligned}
\end{equation*}

А для сигнала без нагрузки:

\begin{equation*}
\begin{aligned}
k_2 &= \frac{V_\Phi}{l} = (3.91 \pm 0.002)~\text{МГц} \\
V_\Phi &= (1.966 \pm 0.02) * 10^{10}~\text{см/с}
\end{aligned}
\end{equation*}

Для АЧХ и ФЧХ
\begin{figure}[h!]
	\centering
	\includegraphics[width=20cm]{a5.PNG}
	\caption{АЧХ и ФЧХ}
	\label{fig:Holl2}
\end{figure}

\end{document}